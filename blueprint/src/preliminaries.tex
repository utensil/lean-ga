\section{Preliminaries}
\label{cha:preliminaries}

This section introduces the algebraic environment of Clifford Algebra,
covering vector spaces, groups, algebras, representations, modules, multilinear algebras,
quadratic forms, filtrations and graded algebras.

The material in this section should be familiar to the reader, but it is worth reading
 through it to become familiar with the notation and terminology that is used,
 as well as their counterparts in Lean, which usually require some additional treatment, both
 mathematically and technically (probably applicable to other formal proof verification systems).

Details can be found in the references in corresponding section, or you may hover a definition/theorem,
then click on L∃∀N for the Lean 4 code.

In this section, we follow \cite{jadczyk2019notes}, with supplements from \cite{garling2011clifford, chen2016infinitely}, 
and modifications to match the counterparts in Lean's \Mathlib.

\begin{remark}
\label{mk:Informal}

We unify the informal mathematical language for a definition to:

\optional{Let \placeholder{$A$} be a \placeholder{\vocab{concept $A$}}. }
A \newvocab{\placeholder{concept $B$}} is a set/pair/triple/tuple $(B, \mathtt{op}, ...)$, satisfying:

\begin{enumerate}
    \item \placeholder{$B$} is a \placeholder{\vocab{concept $C$}} \optional{ over \placeholder{$A$} under \placeholder{op} }.

    \item \placeholder{formula} for all \placeholder{elements in $B$} \optional{(\vocab{ \placeholder{property} })}.
    
    \item \optional{for each \placeholder{element} in \placeholder{\vocab{concept $A$}} }
    there exists \placeholder{element} such that
    \placeholder{formula} for all \placeholder{elements in concept $B$}.

    \item \optional{\placeholder{op} is called \placeholder{\vocab{op name}}, }for all \placeholder{elements in $B$}, we have

        \begin{enumerate}[(i)]

        \item \placeholder{formula}
        \item \placeholder{formula}
        
        \end{enumerate}

        \optional{(\vocab{ \placeholder{property} })}.

\end{enumerate}

By default, \placeholder{$A$} is a set, \placeholder{op} is a binary operation on \placeholder{$A$}.

\end{remark}

\subsection{Basics: from groups to modules}
\label{sec:basics}


In this section, we follow \cite{jadczyk2019notes}, with supplements from \cite{garling2011clifford, chen2016infinitely}, 
and modifications to match the counterparts in Lean's \Mathlib.

\begin{definition}[Group]
    \label{Group}
    \leanok
    \lean{Group, AddGroup}

    A \vocab{group} is a pair $(G, *)$, where $G$ is a set, $*$ is a binary operation, satisfying:

    \begin{enumerate}
    \item $(g * h) * j = g * (h * j)$ for all $g, h, j \in G$ (\vocab{associativity})
    \item there exists $e$ in $G$ such that $e * g = g * e = g$ for all $g \in G$
    \item for each $g \in G$ there exists $g^{-1} \in G$ such that $g * g^{-1} = g^{-1} * g = e$

    \end{enumerate}

\end{definition}

\begin{remark}
    \label{mk:Group}
    
    It then follows that $e$, the \vocab{identity element}, is unique, and that for each $g \in G$ the \vocab{inverse} $g^{-1}$ is unique.

    A group G is abelian, or \vocab{commutative}, if $g * h = h * g$ for all $g, h \in G$.

\end{remark}

\begin{remark}
    \label{mk:Notation}

    In literatures, the binary operation are usually denoted by juxtaposition, and is understood to be a mapping
    $(g, h) \mapsto g * h$ from $G \times G$ to $G$.
    
    \Mathlib uses a slightly different way to encode this, $G \to G \to G$ is understood to be $G \to (G \to G)$,
    that sends $g \in G$ to a mapping that sends $h \in G$ to $g * h \in G$.
    
    Further more, a mathimatical construct is represented by a "type", as Lean has a dependent type theory foundation. %TODO cite
    
    It can be denoted multiplicatively as $*$ in \MathlibDoc{Group}
    or additively as $+$ in \MathlibDoc{AddGroup}, where $e$ will be denoted by $1$ or $0$, respectively.

    Sometimes we use notations with subscripts (e.g. $*_G$, $1_G$) to indicate where they are.

    We will use the corresponding notation in \Mathlib for future operations without further explanation.

\end{remark}

\begin{definition}[Monoid]
    \label{Monoid}
    \leanok
    \lean{Monoid, AddMonoid}

    A \vocab{monoid} is a pair $(R, *)$, satisfying:

    \begin{enumerate}
    \item $(a * b) * c = a * (b * c)$ for all $a, b, c \in R$ (\vocab{associativity}),

    \item There is an element $1 \in R$ such that $1 * a = a * 1 = a$ for all $a$ in $R$
    (that is $1$ is the multiplicative identity (\vocab{neutral element}).

    \end{enumerate}

\end{definition}

\begin{definition}[Ring]
    \label{Ring}
    \leanok
    \lean{Ring, CommRing, CommSemiring}
    \uses{Group, Monoid}

    A \vocab{ring} is a triple $(R, +, *)$, satisfying:

    \begin{enumerate}
    \item The elements of $R$ form a \vocab{commutative group} under $+$;

    \item The elements of $R$ form a \vocab{monoid} under $*$;

    \item If $a, b, c$ are elements of $R$, we have

    $$
    a * (b + c) = a * b + a * c,
    $$

    $$
    (a + b) * c = a * c + b * c
    $$

    (left and right \vocab{distributivity} over $+$).

    \end{enumerate}

\end{definition}

\begin{remark}
    \label{mk:CommRing}

    In applications to Clifford algebras $R$ will be always assumed to be \vocab{commutative}.
    
\end{remark}

\begin{definition}[Division ring]
    \label{DivisionRing}
    \leanok
    \lean{DivisionRing}
    \uses{Ring}

    A ring containing at least two elements, in which
    every nonzero element $a$ has a multiplicative inverse $a^{-1}$ is called a \vocab{division ring} 
    (sometimes also called a "skew field").

\end{definition}

% TODO resume later when needed
% \begin{definition}[Field]
%     \label{Field}
%     \leanok
%     \lean{Field}
%     \uses{Ring}

%     A commutative division ring is called a \vocab{field}.

% \end{definition}

% TODO resume later when needed
% \begin{definition}[Characteristic]
%     \label{ringChar}
%     \leanok
%     \lean{ringChar}
%     \uses{Ring}

%     Let $R$ be a ring with unit element 1 . The \vocab{characteristic} of $R$ is the smallest positive number $n$ such that

%     $$
%     \underbrace{1+\ldots+1}_{n \text { summands }}=0 \text {. }
%     $$
    
%     If such a number does not exist, the characteristic is defined to be 0 .

% \end{definition}

% \begin{remark}
%     \label{mk:characteristic}

%     Equivalently, it can be defined to be the unique $p \in \mathbb{N}$ satisfying:

%     $$
%     \forall x \in \mathbb{N}, x = 0 \iff p \mid x
%     $$

%     where
    
%     \begin{itemize}

%         \item $p \mid x$ is defined as $\exists y \in \mathbb{N}, x = p y$
    
%         % TODO: confirm to what extent it demands the map
%         \item $x = 0$ asks that there exists a map $f : \mathbb{N} \to R$ such that $0 \in \mathbb{N} ↦ 0 \in R$.
        
%     \end{itemize}

%     This is how the characteristic of $R$ is defined in Lean.

% \end{remark}

\begin{definition}[Module]
    \label{Module}
    \leanok
    \lean{Module}
    \uses{Group, Ring}

    Let $R$ be a commutative ring. A \vocab{module} over $R$ (in short $R$-module) is a set $M$ such that

    \begin{enumerate}
      \item M has a structure of an additive group,
    
      \item For every $a, b \in R$, $x, y \in M$, an operation $\bu : R \to M \to M$ called scalar multiplication is defined, and we have
      
        \begin{enumerate}[i]
            \item $a \bu (x + y) = a \bu x + b \bu y$,
            \item $(a + b) \bu x = a \bu x + b \bu x$,
            \item $a * (b \bu x)=(a * b) \bu x$,
            \item $1_R \bu x = x$.
        \end{enumerate}
    \end{enumerate}

\end{definition}

\begin{remark}
    \label{mk:Module}

    The notation of scalar multiplication is generalized as heterogeneous scalar multiplication in \Mathlib:

    $$
    bu : \alpha \to \beta \to \gamma
    $$

    where $\alpha$, $\beta$, $\gamma$ are different types.
    
\end{remark}

\begin{definition}[Vector space]
    \label{VectorSpace}
    \leanok
    \lean{Module}
    \uses{Module, DivisionRing}

    If $R$ is a \vocab{division ring}, then a module $M$ over $R$ is called a \vocab{vector space}.

\end{definition}

\begin{remark}
    \label{mk:VectorSpace}

    For generality, \Mathlib uses \MathlibDoc{Module} throughout for vector spaces,
    particularly, for a vector space $V$, it's usually declared as

    \begin{lstlisting}
        variable [DivisionRing K] [AddCommGroup V] [Module K V]
    \end{lstlisting}

    for definitions/theorems about it, and most of them can be found under \textsf{Mathlib.LinearAlgebra}. % e.g. \MathlibDoc{LinearIndependent}.
    
\end{remark}

\begin{remark}
    \label{mk:Submodule}

    A \vocab{submodule} $N$ of $M$ is a module $N$ such that every element of $N$ is also an element of $M$.

    If $M$ is a vector space then $N$ is called a \vocab{subspace}.

\end{remark}

% TODO resume later when needed
% \begin{definition}[Submodule]
%     \label{Submodule}
%     \leanok
%     \lean{Submodule}
%     \uses{Module}

%     A \vocab{submodule} $N$ of $M$ is a module $N$ such that every element of $N$ is also an element of $M$.

% \end{definition}

% \begin{remark}
%     \label{mk:Submodule}

%     If $M$ is a vector space then $N$ is called a \vocab{subspace}.
    
% \end{remark}

\begin{definition}[Algebra]
    \label{Algebra}
    \leanok
    \lean{Algebra}
    \uses{Module, Ring}

    An \vocab{algebra} $A$ over $R$ is a module over $R$ with a multiplication which makes $A$ a ring and satisfying

    $$
    \alpha(x y)=(\alpha x) y=x(\alpha y),(x, y \in A, \alpha \in R) .
    $$

\end{definition}

\begin{remark}
    \label{mk:Algebra}

    What's simply called algebra is actually associative algebra with identity, a.k.a. \vocab{associative unital algebra}. See
    \href{https://ncatlab.org/nlab/show/red%20herring%20principle}{the red herring principle}
    for more about such phenomenon for naming, particularly the example of (possibly) \vocab{nonassociative algebra}.
    
\end{remark}

\begin{definition}[Ring homomorphism]
    \label{RingHom}
    % \leanok
    \lean{RingHom, RingHomClass}
    \uses{Ring}

    Let $(\alpha, +_\alpha, *_\alpha)$, and $(\beta, +_\beta, *_\beta)$ be rings.
    
    A \vocab{ring homomorphism}, from $\alpha$ to $\beta$ is a function $f : \alpha \to_{+*} \beta$ such that

	\begin{enumerate}[(i)]
		\item $f(x +_{\alpha} y) = f(x) +_{\beta} f(y)$ for each $x,y \in \alpha$.
		\item $f(x *_{\alpha} y) = f(x) *_{\beta} f(y)$ for each $x,y \in \alpha$.
		\item $f(1_{\alpha}) = 1_{\beta}$.
	\end{enumerate}

\end{definition}

\begin{definition}[FreeAlgebra]
    \label{FreeAlgebra}
    % \leanok
    \lean{FreeAlgebra}
    \uses{Ring, Module}

    TODO

\end{definition}

\begin{definition}[LinearMap]
    \label{LinearMap}
    % \leanok
    \lean{LinearMap}
    \uses{Module, RingHom}

    TODO

\end{definition}

\begin{definition}[RingQuot]
    \label{RingQuot}
    % \leanok
    \lean{RingQuot}
    \uses{Module, RingHom}

    TODO

\end{definition}

\begin{definition}[TensorAlgebra relation]
    \label{TensorAlgebra_Rel}
    % \leanok
    \lean{TensorAlgebra.Rel}
    \uses{FreeAlgebra, LinearMap}

    TODO

\end{definition}

\begin{definition}[Tensor algebra]
    \label{TensorAlgebra}
    % \leanok
    \lean{TensorAlgebra}
    \uses{Algebra, FreeAlgebra, TensorAlgebra_Rel, RingQuot}

    Let $M$ be a module over $R$. An algebra $T$ is called a \vocab{tensor algebra} over $M$ (or "of $M$ ")
    if it satisfies the following universal property

    \begin{enumerate}
    \setcounter{enumi}{1}
    \item $T$ is an algebra containing $M$ as a \vocab{submodule}, and it is \vocab{generated by} $M$,
    \item Every linear mapping $\lambda$ of $M$ into an algebra $A$ over $R$, can be extended to 
    a \vocab{homomorphism} $\theta$ of $T$ into $A$.
    \end{enumerate}

\end{definition}

\begin{remark}
    \label{mk:TensorAlgebra}

    The properties above are equivalent to the following:

    \begin{enumerate}
        \setcounter{enumi}{1}
        \item $T$ is the free (associative, unital) $R$-algebra generated by $M$.
        \item additional relations making the inclusion of $M$ into an $R$-linear map
    \end{enumerate}

    As ideals haven't been formalized for the non-commutative case, \Mathlib uses \MathlibDoc{RingQuot} which is
    the quotient of a non-commutative ring by an arbitrary relation.
    
\end{remark}

\subsection{Algebras}
\label{sec:algebras}


\begin{definition}[Ring homomorphism]
    \label{RingHom}
    \leanok
    \lean{RingHom, RingHomClass, algebraMap}
    \uses{Ring}

    Let $(\alpha, +_\alpha, *_\alpha)$ and $(\beta, +_\beta, *_\beta)$ be rings.
    
    A \newvocab{ring homomorphism} from $\alpha$ to $\beta$ is a map $\rfun : \alpha \rmap \beta$ such that

    \begin{enumerate}[(i)]
        \item $\rfun(x +_{\alpha} y) = \rfun(x) +_{\beta} \rfun(y)$ for each $x,y \in \alpha$.
        \item $\rfun(x *_{\alpha} y) = \rfun(x) *_{\beta} \rfun(y)$ for each $x,y \in \alpha$.
        \item $\rfun(1_{\alpha}) = 1_{\beta}$.
    \end{enumerate}

\end{definition}

\begin{remark}
    \label{mk:homomorphism}

    \newvocab{Isomorphism} $A \cong B$ is a bijective \vocab{homomorphism} $\phi : A \to B$
    (it follows that $\phi^{-1} : B \to A$ is also a \vocab{homomorphism}).

    \newvocab{Endomorphism} is a \vocab{homomorphism} from an object to itself, denoted $\operatorname{End}(A)$.

    \newvocab{Automorphism} is an \vocab{endomorphism} which is also an \vocab{isomorphism}, denoted $\operatorname{Aut}(A)$.

\end{remark}

\begin{definition}[Algebra]
    \label{Algebra}
    \leanok
    \lean{Algebra}
    \uses{RingHom, Module}

    Let $R$ be a commutative ring. An \newvocab{algebra} $A$ over $R$ is a pair $(A, \bu)$, satisfying:

    \begin{enumerate}
    \item $A$ is a \vocab{ring} under $*$.
    
    \item there exists a \vocab{ring homomorphism} from $R$ to $A$, denoted $\rfun : R \to_{+*} A$.
    
    \item $\bu : R \to M \to M$ is a \vocab{scalar multiplication}
    
    \item for every $r \in R$, $x \in A$, we have

    \begin{enumerate}[(i)]
        \item $r * x = x * r$ (commutativity between $R$ and $A$)
        \item $r \bu x = r * x$ (definition of scalar multiplication)
    \end{enumerate}

    \end{enumerate}

    where we omitted that the ring homomorphism $\rfun$ is applied to $r$ before each multiplication.

\end{definition}

\begin{remark}
    \label{mk:AlgebraNotation}

    Following literatures, for $r \in R$, 
    usually we write $\rfun_A(r) : R \to_{+*} A$ as a product $r \rfun_A$ if not omitted,
    while they are written as a call to
    \ifuselistings
    \lstinline|algebraMap _ _ r| in \Mathlib,
    which is defined to be \lstinline|Algebra.toRingHom r|.
    \else
    \mintinline{lean4}{algebraMap _ _ r} in \Mathlib,
    which is defined to be \mintinline{lean4}{Algebra.toRingHom r}.
    \fi

\end{remark}

\begin{remark}
    \label{mk:AlgebraLiterature}

    The definition above (adopted in \Mathlib) is more general than the definition in literature:

    Let $R$ be a commutative ring. An \newvocab{algebra} $A$ over $R$ is a pair $(M, *)$, satisfying:

    \begin{enumerate}
    \item $A$ is a \vocab{module} $M$ over $R$ under $+$ and $\bu$.

    \item $A$ is a \vocab{ring} under $*$.

    \item For $x, y \in A, a \in R$, we have
    
    $$
    a \bu (x * y) = (a \bu x) * y = x * (a \bu y)
    $$

    \end{enumerate}

    See \emph{Implementation notes} in \MathlibDoc{Algebra} for details.
    
\end{remark}

\begin{remark}
    \label{mk:AlgebraName}

    What's simply called algebra is actually associative algebra with identity, a.k.a. \vocab{associative unital algebra}. See
    \href{https://ncatlab.org/nlab/show/red%20herring%20principle}{the red herring principle}
    for more about such phenomenon for naming, particularly the example of (possibly) \vocab{nonassociative algebra}.
    
\end{remark}

\begin{definition}[Algebra homomorphism]
    \label{AlgHom}
    \leanok
    \lean{AlgHom}
    \uses{RingHom}

    Let $A$ and $B$ be $R$-algebras. $\rfun_A$ and $\rfun_B$ are \vocab{ring homomorphisms} from $R$ to $A$ and $B$, respectively.

    A \newvocab{algebra homomorphism} from $A$ to $B$ is a map $f : \alpha \amap \beta$ such that

    \begin{enumerate}

    \item $f$ is a \vocab{ring homomorphism}

    \item $f(\rfun_{A}(r)) = \rfun_{B}(r)$ for each $r \in R$

    \end{enumerate}

\end{definition}

\begin{definition}[Ring quotient]
    \label{RingQuot}
    \leanok
    \lean{RingQuot}
    \uses{Module, RingHom}

    Let $R$ be a non-commutative ring, $r$ an arbitrary equivalence relation on $R$.
    The \newvocab{ring quotient} of $R$ by $r$
    is the quotient of $R$ by the strengthen equivalence relation of $r$
    such that for all $a, b, c$ in $R$:

    \begin{enumerate}

    \item $a + c \sim b + c$ if $a \sim b$
    \item $a * c \sim b * c$ if $a \sim b$
    \item $a * b \sim a * c$ if $b \sim c$
    
    \end{enumerate}

    i.e. the equivalence relation is compatible with the ring operations $+$ and $*$.

\end{definition}

\begin{remark}
    \label{mk:RingQuot}

    As ideals haven't been formalized for the non-commutative case, \Mathlib uses \MathlibDoc{RingQuot} in places
    where the quotient of non-commutative rings by ideal is needed.

    The universal properties of the quotient are proven, and should be used instead of the definition that is subject to change.
    
\end{remark}

\begin{definition}[Free algebra]
    \label{FreeAlgebra}
    \leanok
    \lean{FreeAlgebra, FreeAlgebra.Pre, FreeAlgebra.Rel}
    \uses{Algebra, RingQuot}

    Let $X$ be an arbitrary set.
    An \newvocab{free $R$-algebra} $A$ on $X$ (or ``\newvocab{generated by} $X$ ") is the \vocab{ring quotient} of the following inductively constructed set $A_{\pre}$

    \begin{enumerate}

    \item for all $x$ in $X$, there exists a map $X \to A_{\pre}$.
    \item for all $r$ in $R$, there exists a map $R \to A_{\pre}$.
    \item for all $a, b$ in $A_{\pre}$, $a + b$ is in $A_{\pre}$.
    \item for all $a, b$ in $A_{\pre}$, $a * b$ is in $A_{\pre}$.
    
    \end{enumerate}

    by that equivalence relation that makes $A$ an \vocab{$R$-algebra}, namely:

    \begin{enumerate}
    
    \item there exists a \vocab{ring homomorphism} from $R$ to $A_{\pre}$, denoted $R \to_{+*} A_{\pre}$.
    \item $A$ is a \vocab{commutative group} under $+$.
    \item $A$ is a \vocab{monoid} under $*$.
    \item left and right \vocab{distributivity} under $*$ over $+$.
    \item $0 * a \sim a * 0 \sim 0$.
    \item for all $a, b, c$ in $A$, if $a \sim b$, we have
    
    \begin{enumerate}[(i)]
    
    \item $a + c \sim b + c$
    \item $c + a \sim c + b$
    \item $a * c \sim b * c$
    \item $c * a \sim c * b$

    \end{enumerate}

    (\vocab{compatibility} with the ring operations $+$ and $*$)

    \end{enumerate}

\end{definition}

\begin{remark}
    \label{mk:FreeAlgebra}

    What we defined here is the \newvocab{free (associative, unital) $R$-algebra} on $X$, it can be denoted $R\langle X \rangle$,
    expressing that it's freely generated by $R$ and $X$, where $X$ is the set of generators.

\end{remark}

\begin{definition}[Linear map]
    \label{LinearMap}
    \leanok
    \lean{LinearMap}
    \uses{Module, RingHom}

    Let $R, S$ be rings, $M$ an $R$-module, $N$ an $S$-module.
    A \newvocab{linear map} from $M$ to $N$ is a function $f : M \to_{l} N$ over a ring homomorphism $\sigma : R \to_{+*} S$, satisfying:

    \begin{enumerate}

    \item $f(x + y) = f(x) + f(y)$ for all $x, y \in M$.
    \item $f(r \bu x) = \sigma(r) \bu f(x)$ for all $r \in R$, $x \in M$.
    
    \end{enumerate}

\end{definition}

\begin{remark}
    \label{mk:linearMap}

    The set of all linear maps from $M$ to $M'$ is denoted $\operatorname{Lin}(M, M')$,
    and $\operatorname{Lin}(M)$ for mapping from $M$ to itself.
    
    $\operatorname{Lin}(M)$ is an \vocab{endomorphism}.

\end{remark}

\begin{definition}[Tensor algebra]
    \label{TensorAlgebra}
    \leanok
    \lean{TensorAlgebra}
    \uses{FreeAlgebra, LinearMap, RingQuot}

    % Let $R$ be a commutative ring, $M$ a $R$-module.
    Let $A$ be a \vocab{free $R$-algebra} generated by module $M$, let $\iota : M \to A$ denote the map from $M$ to $A$.

    An \newvocab{tensor algebra} over $M$ (or ``of $M$ ") $T$ is the \vocab{ring quotient} of the \vocab{free $R$-algebra} generated by $M$, 
    by the equivalence relation satisfying:

    \begin{enumerate}

    \item for all $a, b$ in $M$, $\iota(a + b) \sim \iota(a) + \iota(b)$.
    \item for all $r$ in $R$, $a$ in $M$, $\iota(r \bu a) \sim r * \iota(a)$.
    
    \end{enumerate}

    i.e. making the inclusion of $M$ into an \vocab{$R$-linear map}.

\end{definition}

\begin{remark}
    \label{mk:TensorAlgebra}

    The definition above is equivalent to the following definition in literature:

    Let $M$ be a module over $R$. An algebra $T$ is called a \vocab{tensor algebra} over $M$ (or ``of $M$ ")
    if it satisfies the following universal property

    \begin{enumerate}
    \item $T$ is an algebra containing $M$ as a \vocab{submodule}, and it is \vocab{generated by} $M$,
    \item Every linear mapping $\lambda$ of $M$ into an algebra $A$ over $R$, can be extended to 
    a \vocab{homomorphism} $\theta$ of $T$ into $A$.
    \end{enumerate}
    
\end{remark}

\subsection{Forms}
\label{sec:forms}

\begin{definition}[Bilinear form]
    \label{BilinForm}
    \leanok
    \lean{BilinForm}
    \uses{Module}

    Let $R$ be a ring, $M$ an $R$-module. An \vocab{bilinear form} $B$ over $M$ is a map $B : M \to M \to R$, satisfying:

    \begin{enumerate}

    \item $ B(x + y, z) = B(x, z) +B(y, z) $
    
    \item $ B(x, y + z) = B(x, y) +B(x, z) $
    
    \item $ B(a \bu x, y) = a * B(x, y)$
    
    \item $ B(x, a \bu y) = a * B(x, y)$
        
    \end{enumerate}

    for all $a \in R, x, y, z \in M$.

\end{definition}

\begin{definition}[Quadratic form]
    \label{QuadraticForm}
    \leanok
    \lean{QuadraticForm}
    \uses{BilinForm}

    Let $R$ be a commutative ring, $M$ a $R$-module. An \vocab{quadratic form} $Q$ over $M$ is a map $Q : M \to R$, satisfying:

    \begin{enumerate}

    \item $ Q(a \bu x) = a * a * Q(x)$ for all $a \in R, x \in M$.
    
    \item there exists a companion \vocab{bilinear form} $B : M \to M \to R$, such that $Q(x + y) = Q(x) + Q(y) + B(x, y)$
    
    \end{enumerate}
\end{definition}
