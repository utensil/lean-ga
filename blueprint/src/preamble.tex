\usepackage[usenames,svgnames,dvipsnames]{xcolor}
\usepackage{amsfonts, amsthm, amssymb, amsmath}
\usepackage{tikz-cd}
\usetikzlibrary{decorations}
% \usepackage{color}
\usepackage[unicode]{hyperref}
\hypersetup{%
  colorlinks=true,
%   linkcolor=blue,
  urlcolor=blue
}
% NOTE: Won't work in web version for roman numerals, but it works well for PDF version
\usepackage{enumerate}% http://ctan.org/pkg/enumerate

\usepackage{listings}
\usepackage[utf8]{inputenc}
\usepackage[T1]{fontenc}
\usepackage{newunicodechar}
\usepackage{lean_unicode}

% \usepackage[frozencache,cachedir=minted-cache]{minted}
% \usepackage{minted}
% \setminted{
%   fontsize=\footnotesize,
%   frame=single,
%   breaklines
% }
% \setmintedinline{fontsize=\currentfontsize}
% \definecolor{bg}{rgb}{0.95,0.95,0.95}
% \newmintinline[leanc]{theorem.py:LeanLexer -x}{bgcolor=bg}
% \newminted[leancode]{theorem.py:LeanLexer -x}{}
% % https://tex.stackexchange.com/a/276750/41112
% \newlength{\mintedfboxsep}
% \setlength{\mintedfboxsep}{1.5pt}

\date{\today}

\setcounter{tocdepth}{1}

% handle Lean code highlighting following https://lean-lang.org/lean4/doc/syntax_highlight_in_latex.html
\definecolor{keywordcolor}{rgb}{0.7, 0.1, 0.1}   % red
\definecolor{tacticcolor}{rgb}{0.0, 0.1, 0.6}    % blue
\definecolor{commentcolor}{rgb}{0.4, 0.4, 0.4}   % grey
\definecolor{symbolcolor}{rgb}{0.0, 0.1, 0.6}    % blue
\definecolor{sortcolor}{rgb}{0.1, 0.5, 0.1}      % green
\definecolor{attributecolor}{rgb}{0.7, 0.1, 0.1} % red

\def\lstlanguagefiles{lstlean.tex}
% set default language
\lstset{language=lean}

\ifplastex
% begin web specific preambles
% end web specific preambles
\else
% begin print specific preambles

% Somehow this renders to plain text without brackets, makes it indistinguishable from the surrounding text.
% And it doesn't work with plastex, the citation is not rendered at all for the web version.
% \usepackage{natbib}
% \citestyle{acmauthoryear}

\usepackage{unicode-math}
\usepackage{fontspec}
\setmathfont{latinmodern-math.otf}
\setmathfont[range=\varnothing]{Asana-Math.otf}
\setlength{\textwidth}{6.5in}
\setlength{\oddsidemargin}{-0.1in}
\setlength{\evensidemargin}{-0.1in}
% end print specific preambles
\fi

%%fakesection Defining theorem styles

\ifplastex
% theorem styles are implemented in style.css for plastex i.e. website
% here we simply use the default styles
\else
% theorem styles are implemented by package `mdframed` for LaTeX i.e. PDF
% mostly follows https://github.com/vEnhance/napkin/blob/main/tex/preamble.tex
\usepackage{thmtools}
\usepackage[framemethod=TikZ]{mdframed}

\theoremstyle{definition}
\mdfdefinestyle{mdbluebox}{%
	roundcorner = 10pt,
	linewidth=1pt,
	skipabove=12pt,
	innerbottommargin=9pt,
	skipbelow=2pt,
	nobreak=true,
	linecolor=blue,
	backgroundcolor=TealBlue!5,
}
\declaretheoremstyle[
	headfont=\sffamily\bfseries\color{MidnightBlue},
	mdframed={style=mdbluebox},
	headpunct={\\[3pt]},
	postheadspace={0pt}
]{thmbluebox}

\mdfdefinestyle{mdredbox}{%
	linewidth=0.5pt,
	skipabove=12pt,
	frametitleaboveskip=5pt,
	frametitlebelowskip=0pt,
	skipbelow=2pt,
	frametitlefont=\bfseries,
	innertopmargin=4pt,
	innerbottommargin=8pt,
	nobreak=true,
	linecolor=RawSienna,
	backgroundcolor=Salmon!5,
}
\declaretheoremstyle[
	headfont=\bfseries\color{RawSienna},
	mdframed={style=mdredbox},
	headpunct={\\[3pt]},
	postheadspace={0pt},
]{thmredbox}

\mdfdefinestyle{mdgreenbox}{%
	skipabove=8pt,
	linewidth=2pt,
	rightline=false,
	leftline=true,
	topline=false,
	bottomline=false,
	linecolor=ForestGreen,
	backgroundcolor=ForestGreen!5,
}
\declaretheoremstyle[
	headfont=\bfseries\sffamily\color{ForestGreen!70!black},
	bodyfont=\normalfont,
	spaceabove=2pt,
	spacebelow=1pt,
	mdframed={style=mdgreenbox},
	headpunct={ --- },
]{thmgreenbox}
\declaretheoremstyle[
	headfont=\bfseries\sffamily\color{ForestGreen!70!black},
	bodyfont=\normalfont,
	spaceabove=2pt,
	spacebelow=1pt,
	mdframed={style=mdgreenbox},
	headpunct={},
]{thmgreenbox*}

\mdfdefinestyle{mdblackbox}{%
	skipabove=8pt,
	linewidth=3pt,
	rightline=false,
	leftline=true,
	topline=false,
	bottomline=false,
	linecolor=black,
	backgroundcolor=RedViolet!5!gray!5,
}
\declaretheoremstyle[
	headfont=\bfseries,
	bodyfont=\normalfont\small,
	spaceabove=0pt,
	spacebelow=0pt,
	mdframed={style=mdblackbox}
]{thmblackbox}
\fi

%%fakesection Applying theorem styles

\numberwithin{equation}{subsection}

\ifplastex

% \theoremstyle{plain}

\newtheorem{theorem}{Theorem}
\numberwithin{theorem}{subsection}

\newtheorem*{theoremx}{Theorem} % theorem not in dep graph
\newtheorem{lemma}[theorem]{Lemma}
\newtheorem{lemmax}[theorem]{Lemma} % lemma not in dep graph
\newtheorem{corollary}[theorem]{Corollary}
\newtheorem{proposition}[theorem]{Proposition}
% \newtheorem{conjecture}[theorem]{Conjecture}
% \newtheorem{defprop}[theorem]{Definition/Proposition}

% \newtheorem{problem}[theorem]{Problem}
% \newtheorem{assumption}[theorem]{Assumption}

\theoremstyle{definition}
\newtheorem{definition}[theorem]{Definition}
% \newtheorem{exercise}[theorem]{Exercise}
% \newtheorem{warning}[theorem]{Warning}
% \newtheorem{question}[theorem]{Question}
% \newtheorem{observation}[theorem]{Observation}

\theoremstyle{definition}
\newtheorem{example}[theorem]{Example}
\newtheorem{examples}[theorem]{Examples}

\newtheorem{remark}[theorem]{Remark}
\newtheorem*{remarkx}{Remark} % unnumbered remark

\else

\theoremstyle{definition}
\declaretheorem[style=thmbluebox,name=Theorem,numberwithin=subsection]{theorem}
\declaretheorem[style=thmbluebox,name=Theorem,numberwithin=subsection]{theoremx} % theorem not in dep graph
\declaretheorem[style=thmbluebox,name=Lemma,sibling=theorem]{lemma}
\declaretheorem[style=thmbluebox,name=Lemma,sibling=theorem]{lemmax} % lemma not in dep graph
\declaretheorem[style=thmbluebox,name=Corollary,sibling=theorem]{corollary}
\declaretheorem[style=thmbluebox,name=Proposition,sibling=theorem]{proposition}

\theoremstyle{definition}
% \newtheorem{claim}[theorem]{Claim}
\newtheorem{definition}[theorem]{Definition}
% \newtheorem{fact}[theorem]{Fact}
% \newtheorem{abuse}[theorem]{Abuse of Notation}

\declaretheorem[style=thmredbox,name=Example,sibling=theorem]{example}

\theoremstyle{theorem}
% \declaretheorem[name=Question,sibling=theorem,style=thmblackbox]{ques}
% \declaretheorem[name=Exercise,sibling=theorem,style=thmblackbox]{exercise}
\declaretheorem[name=Remark,sibling=theorem,style=thmgreenbox]{remark}
\declaretheorem[name=Remark,sibling=theorem,style=thmgreenbox*]{remark*}
% \declaretheorem[name=Step,style=thmgreenbox]{step} % only used in Lebesgue int
\fi
