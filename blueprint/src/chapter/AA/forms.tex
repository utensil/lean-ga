\begin{definition}[Bilinear form]
    \label{BilinForm}
    \leanok
    \lean{BilinForm}
    \uses{Module}

    Let $R$ be a ring, $M$ an $R$-module. An \newvocab{bilinear form} $B$ over $M$ is a map $B : M \to M \to R$, satisfying:

    \begin{enumerate}

    \item $ B(x + y, z) = B(x, z) +B(y, z) $
    
    \item $ B(x, y + z) = B(x, y) +B(x, z) $
    
    \item $ B(a \bu x, y) = a * B(x, y)$
    
    \item $ B(x, a \bu y) = a * B(x, y)$
        
    \end{enumerate}

    for all $a \in R, x, y, z \in M$.

\end{definition}

\begin{definition}[Quadratic form]
    \label{QuadraticForm}
    \leanok
    \lean{QuadraticForm,QuadraticForm.polar}
    \uses{BilinForm}

    Let $R$ be a commutative ring, $M$ a $R$-module. An \newvocab{quadratic form} $Q$ over $M$ is a map $Q : M \to R$, satisfying:

    \begin{enumerate}

    \item $ Q(a \bu x) = a * a * Q(x)$ for all $a \in R, x \in M$.
    
    \item there exists a companion \vocab{bilinear form} $B : M \to M \to R$, such that $Q(x + y) = Q(x) + Q(y) + B(x, y)$
    
    \end{enumerate}

    In some literatures, the bilinear form is denoted $\Phi$, and called the \newvocab{polar form} associated with the quadratic form $Q$,
    or simply the polar form of $Q$.

\end{definition}

\begin{remark}
    \label{mk:QuadraticForm}

    This notion generalizes to commutative semirings using the approach in \cite{izhakian2016supertropical}.
    
\end{remark}