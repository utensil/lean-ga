Same as the previous section, let $M$ be a module over a commutative ring $R$, equipped with a quadratic form $Q: M \to R$.

We also use $m$ or $m_1, m_2, \dots$ to denote elements of $M$, i.e. vectors, and $x, y, z$ to denote elements of $\Cl(Q)$.

\begin{definition}[Grade involution]
    \label{involute}
    \lean{CliffordAlgebra.involute}
    \leanok
    \uses{iota}

    \newvocab{Grade involution}, intuitively, is negating each basis vector.

    Formally, it's an \vocab{algebra homomorphism} $\alpha : \Cl(Q) \amap \Cl(Q)$, satisfying:

    \begin{enumerate}

    \item $\alpha \circ \alpha = \operatorname{id}$
    
    \item $\alpha(\iota(m)) = - \iota(m)$
    
    \end{enumerate}

    for all $m \in M$.

    It's called \newvocab{main involution} $\alpha$ or \newvocab{main automorphism} in \cite{jadczyk2019notes}, 
    the \newvocab{canonical automorphism} in \cite{gallier2008clifford}.

    It's denoted $\hat{m}$ in \cite{lounestoCliffordAlgebrasSpinors2001}, $\alpha(m)$ in \cite{jadczyk2019notes}, $m^*$ in \cite{chisolm2012geometric}.

    \begin{figure}[H]
    \centering
    \begin{tikzcd}[column sep=huge, row sep=huge]
    \Cl(Q) \arrow[r, "\alpha"] & \Cl(Q) \\
    V \arrow[ru, "-\iota"] \arrow[u, "\iota"]
    \end{tikzcd}
    \end{figure}

\end{definition}

\begin{definition}[Grade reversion]
    \label{reverse}
    \lean{CliffordAlgebra.reverse}
    \leanok
    \uses{iota}

    \newvocab{Grade reversion}, intuitively, is reversing the multiplication order of basis vectors.
    
    Formally, it's an \vocab{algebra homomorphism} $\tau : \Cl(Q) \amap \Cl(Q)^{\mathtt{op}}$, satisfying:

    \begin{enumerate}

    \item $\tau(m_1 m_2) = \tau(m_2) \tau(m_1)$
    \item $\tau \circ \tau = \operatorname{id}$
    \item $\tau(\iota(m)) = \iota(m)$
    
    \end{enumerate}

    It's called \newvocab{anti-involution} $\tau$ in \cite{jadczyk2019notes}, the \newvocab{canonical anti-automorphism} in \cite{gallier2008clifford},
    also called \newvocab{transpose}/\newvocab{transposition} in some literature, following tensor algebra or matrix.

    It's denoted $\tilde{m}$ in \cite{lounestoCliffordAlgebrasSpinors2001}, $m^\tau$ in \cite{jadczyk2019notes} (with variants like $m^t$ or $m^\top$ in other literatures), $m^\dagger$ in \cite{chisolm2012geometric}.

    \begin{figure}[H]
    \centering
    \begin{tikzcd}[column sep=huge, row sep=huge]
    \Cl(Q) \arrow[r, "\tau"] & \Cl(Q)^{\mathtt{op}} \\
    V \arrow[ru, "\iota"] \arrow[u, "\iota"]
    \end{tikzcd}
    \end{figure}

\end{definition}

\begin{definition}[Clifford conjugate]
    \label{conjugate}
    \lean{CliffordAlgebra.reverse}
    \leanok
    \uses{involute,reverse}

    \newvocab{Clifford conjugate} is an \vocab{algebra homomorphism}  ${*} : \Cl(Q) \amap \Cl(Q)$,
    denoted $x^{*}$ (or even $x^\dagger$, $xᘁ$ in some literatures),
    defined to be:

    $$ x^{*} = \operatorname{reverse}(\operatorname{involute}(x)) = \tau(\alpha(x)) $$ for all $x \in \Cl(Q)$, satisfying
    (as a \href{https://en.wikipedia.org/wiki/*-algebra#*-ring}{\newvocab{$*$-ring}}):

    \begin{enumerate}

    \item $(x + y)^{*} = (x)^{*} + (y)^{*}$
    \item $(x y)^{*} = (y)^{*} (x)^{*}$
    \item ${*} \circ {*} = \operatorname{id}$
    \item $1^{*} = 1$
    
    \end{enumerate}

    and (as a \href{https://en.wikipedia.org/wiki/*-algebra#*-algebra}{\newvocab{$*$-algebra}}):

    $$ (r x)^{*} = r' x^{*} $$
    
    for all $r \in R$, $x, y \in \Cl(Q)$ where $'$ is the involution of the commutative $*$-ring $R$.

    Note: In our current formalization in \Mathlib, the application of the involution on $r$ is ignored,
    as there appears to be nothing in the literature that advocates doing this.

    % Grade reversion, reversing the multiplication order of basis vectors.
    % Also called *transpose* in some literature, thus denoted 

    % It's called \newvocab{anti-involution} $\tau$ in \cite{jadczyk2019notes}.

    \vocab{Clifford conjugate} is denoted $\bar{m}$ in \cite{lounestoCliffordAlgebrasSpinors2001} and most literatures, $m^\ddagger$ in \cite{chisolm2012geometric}.

\end{definition}

\begin{definition}[$Z_2$-graded derivations $i_f$, anti-derivation]
    \label{antiDeriv}
    % \lean{}
    % \leanok
    \uses{Dual,CliffordAlgebra}

    We denote by $\operatorname{End}(M)$ the algebra of all \vocab{endomorphisms} (linear maps) of $M$.
    
    For $m \in M$, the linear operator
    $e_{m} \in$ $\operatorname{End}(\mathrm{T}(M)), \mathrm{T}^{p}(M) \rightarrow \mathrm{T}^{p+1}(M)$ is of left multiplication by $m$ :

    $$
    e_{m}: x \mapsto e_{m}(x) = m \otimes x
    $$

    for all $x \in \mathrm{T}(M)$.

    Let $f$ be an element of the \vocab{dual module} $M^{*}$.

    The \newvocab{anti-derivation} $i_{f} : T(M) \lmap T(M)$ is a linear map from $T(M)$ to $T(M)$, satisfying:

    \begin{enumerate}

    \item $i_{f}(1) = 0$
    
    \item $e_{m} \circ i_{f} + i_{f} \circ e_{m} = f(m) \cdot 1$ for all $m \in M$
    
    \end{enumerate}

    The map $f \mapsto i_{f}$ from $M^{*}$ to \vocab{linear transformations} on $T(M)$ is linear. We have

    \begin{enumerate}

    \item $i_{f}(m \otimes x) = f(m) x - m \otimes i_{f}(x)$ for all $m \in M \subset T(M)$, $x \in T(M)$
    
    \item $i_{f}\left( T^{p} M \right) \subset T^{p-1} M$
    
    \item $i_{f}^{2} = 0$
    
    \item $i_{f} i_{g} + i_{g} i_{f} = 0$, for all $f, g \in M^{*}$
    
    \end{enumerate}

    For $m_{1}, \ldots, m_{p} \in M$ we have

    $$
    i_{f}\left(m_{1} \otimes \cdots \otimes m_{p}\right)=\sum_{i=1}^{p}(-1)^{i-1} f\left(m_{i}\right) m_{1} \otimes \cdots \otimes \check{m}_{i} \otimes \cdots \otimes m_{p}
    $$

    where $\check{m}_{i}$ denotes \newvocab{deletion} (of $m_i$ from the multiplication).

    For a quadratic form $Q$ on $M$,
    % the ideal $J(Q)$ is stable under $i_{f}$, i.e. $i_{f}(J(Q)) \subset J(Q)$,
    $\bar{i}_{f}$ can be defined on the quotient Clifford algebra:

    $$
    \iota \circ i_{f} = \bar{i}_{f} \circ \iota,
    $$

    satisfying:

    \begin{enumerate}

    \item $\bar{i}_{f}(1) = 0$ for $1 \in \Cl(Q)$

    \item $\bar{i}_{f}\left( \iota(m) x \right) = f(m) x - \iota(m) \bar{i}_{f}(x)$ for all  $m \in M$, $x \in \Cl(Q)$
    
    \end{enumerate}

    Let $F$ be a bilinear form on $M$. Then every $m \in M$ determines a linear form $f_{m}$ on $M$ defined as $f_{m}(m') = F(m, m')$.
    
    We will denote by $i_{m}^{F}$ the antiderivation $i_{f_{m}}$. We have:

    \begin{enumerate}

    \item $i_{m}^{F}(1) = 0$,

    \item $ i_{m}^{F}(m' \otimes x) = F(m, m') x - m' \otimes i_{m}^{F}(x) $ for all $m' \in M, x \in T(M)$

    \end{enumerate}

    For $x_{1}, \ldots, x_{n}$ in $T(M)$, we have

    $$
    i_{m}^{F}\left(x_{1} \otimes \cdots \otimes x_{n}\right) = \sum_{j=1}^{n}(-1)^{j-1} F\left(m, x_{j}\right) x_{1} \otimes \cdots \otimes \check{x}_{j} \otimes \cdots \otimes x_{n}
    $$

    As it was in the case with $\bar{i}_{f}$, we will denote by $\bar{i}_{x}^{F}$ the antiderivation $\bar{i}_{f}$ for $f_m(m') = F(m, m')$ :

    $$
    \bar{i}_{m}^{F}=\bar{i}_{f}
    $$

    for $f_m(m') = F(m, m'), (m, m' \in M)$

    % TODO: 

    % $f \in V^*$ already at the level of the tensor algebra $T(V)$.

    % $i_f$ is defined recursively through $i_f(1) = 0$, $i_f(x \otimes u) = f(x)u - x \otimes i_f(u)$ for $x \in V, u \in T^p(V)$.

    This is the approach used in \cite{bourbaki2007}, and re-introduced in \cite{jadczyk2019notes,jadczyk2023bundle}.

    % The construction of natural linear mappings $\lambda_F : T(V) → T(V)$ that
    % map every two-sided ideal $I(Q)$ to $I(Q + Q_F)$, where, for any bilinear form $F \in \operatorname{Bil}(V)$,
    % $Q_F$ is the quadratic form $Q_F(x) = F(x, x)$.

    % Following Bourbaki's convention, the descendants of these operators, acting at the level of Clifford algebras, with a bar.

    $\bar{i_f}$ is denoted $\partial_v$ for $v \in \Cl(Q)_1$ in \cite{lundholm2009clifford}.

    This is closely related to \vocab{contraction} (i.e. $\iota(m)\rfloor x = m \rfloor_{F} x \doteq \bar{i}_{m}^{F}(x)$ for $Q = 0$ )
    and \href{https://en.wikipedia.org/wiki/Interior_product}{\vocab{interior product}}.

\end{definition}