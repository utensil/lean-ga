
\begin{definition}[Ring homomorphism]
    \label{RingHom}
    \leanok
    \lean{RingHom, RingHomClass, algebraMap}
    \uses{Ring}

    Let $(\alpha, +_\alpha, *_\alpha)$ and $(\beta, +_\beta, *_\beta)$ be rings.
    
    A \newvocab{ring homomorphism} from $\alpha$ to $\beta$ is a map $\rfun : \alpha \rmap \beta$ such that

    \begin{enumerate}[(i)]
        \item $\rfun(x +_{\alpha} y) = \rfun(x) +_{\beta} \rfun(y)$ for each $x,y \in \alpha$.
        \item $\rfun(x *_{\alpha} y) = \rfun(x) *_{\beta} \rfun(y)$ for each $x,y \in \alpha$.
        \item $\rfun(1_{\alpha}) = 1_{\beta}$.
    \end{enumerate}

\end{definition}

\begin{definition}[Algebra]
    \label{Algebra}
    \leanok
    \lean{Algebra}
    \uses{RingHom, Module}

    Let $R$ be a commutative ring. An \newvocab{algebra} $A$ over $R$ is a pair $(A, \bu)$, satisfying:

    \begin{enumerate}
    \item $A$ is a \vocab{ring} under $*$.
    
    \item there exists a \vocab{ring homomorphism} from $R$ to $A$, denoted $\iota : R \to_{+*} A$.
    
    \item $\bu : R \to M \to M$ is a \vocab{scalar multiplication}
    
    \item for every $r \in R$, $x \in A$, we have

    \begin{enumerate}[(i)]
        \item $r * x = x * r$ (commutativity between $R$ and $A$)
        \item $r \bu x = r * x$ (definition of scalar multiplication)
    \end{enumerate}

    \end{enumerate}

    where we omitted that the ring homomorphism $\iota$ is applied to $r$ before each multiplication,
    while they are explicitly written as \lstinline|toRingHom(r)| in \Mathlib.

\end{definition}

\begin{remark}
    \label{mk:AlgebraLiterature}

    The definition above (adopted in \Mathlib) is more general than the definition in literature:

    Let $R$ be a commutative ring. An \newvocab{algebra} $A$ over $R$ is a pair $(M, *)$, satisfying:

    \begin{enumerate}
    \item $A$ is a \vocab{module} $M$ over $R$ under $+$ and $\bu$.

    \item $A$ is a \vocab{ring} under $*$.

    \item For $x, y \in A, a \in R$, we have
    
    $$
    a \bu (x * y) = (a \bu x) * y = x * (a \bu y)
    $$

    \end{enumerate}

    See \emph{Implementation notes} in \MathlibDoc{Algebra} for details.
    
\end{remark}

\begin{remark}
    \label{mk:AlgebraName}

    What's simply called algebra is actually associative algebra with identity, a.k.a. \vocab{associative unital algebra}. See
    \href{https://ncatlab.org/nlab/show/red%20herring%20principle}{the red herring principle}
    for more about such phenomenon for naming, particularly the example of (possibly) \vocab{nonassociative algebra}.
    
\end{remark}

\begin{definition}[Algebra homomorphism]
    \label{AlgHom}
    \leanok
    \lean{AlgHom}
    \uses{RingHom}

    Let $A$ and $B$ be $R$-algebras. $\rfun_A$ and $\rfun_B$ are \vocab{ring homomorphisms} from $R$ to $A$ and $B$, respectively.
    
    A \newvocab{algebra homomorphism} from $A$ to $B$ is a map $\afun : \alpha \amap \beta$ such that

    $\afun(\rfun_{A}(r)) = \rfun_{B}(r)$ for each $r \in R$.

\end{definition}

\begin{definition}[Quotient of non-commutative ring]
    \label{RingQuot}
    \leanok
    \lean{RingQuot}
    \uses{Module, RingHom}

    Let $R$ be a non-commutative ring, $r$ an arbitrary equivalence relation on $R$.
    The \newvocab{ring quotient} of $R$ by $r$
    is by the strengthen equivalence relation of $r$ such that for all $a, b, c$ in $R$:

    \begin{enumerate}

    \item $a + c \sim b + c$ if $a \sim b$
    \item $a * c \sim b * c$ if $a \sim b$
    \item $a * b \sim a * c$ if $b \sim c$
    
    \end{enumerate}

    i.e. the equivalence relation is compatible with the ring operations $+$ and $*$.

\end{definition}

\begin{remark}
    \label{mk:RingQuot}

    As ideals haven't been formalized for the non-commutative case, \Mathlib uses \MathlibDoc{RingQuot} in places
    where the quotient of non-commutative rings by ideal is needed.

    The universal properties of the quotient are proven, and should be used instead of the definition that is subject to change.
    
\end{remark}

\begin{definition}[Free algebra]
    \label{FreeAlgebra}
    \leanok
    \lean{FreeAlgebra, FreeAlgebra.Pre, FreeAlgebra.Rel}
    \uses{Algebra, RingQuot}

    Let $X$ be an arbitrary set.
    An \newvocab{free $R$-algebra} $A$ on $X$ (or ``\newvocab{generated by} $X$ ") is the \vocab{ring quotient} of the following inductively constructed set $A_{pre}$

    \begin{enumerate}

    \item for all $x$ in $X$, there exists a map $X \to A_{pre}$.
    \item for all $r$ in $R$, there exists a map $R \to A_{pre}$.
    \item for all $a, b$ in $A_{pre}$, $a + b$ is in $A_{pre}$.
    \item for all $a, b$ in $A_{pre}$, $a * b$ is in $A_{pre}$.
    
    \end{enumerate}

    by that equivalence relation that makes $A$ an \vocab{$R$-algebra}, namely:

    \begin{enumerate}
    
    \item there exists a \vocab{ring homomorphism} from $R$ to $A_{pre}$, denoted $R \to_{+*} A_{pre}$.
    \item $A$ is a \vocab{commutative group} under $+$.
    \item $A$ is a \vocab{monoid} under $*$.
    \item left and right \vocab{distributivity} under $*$ over $+$.
    \item $0 * a \sim a * 0 \sim 0$.
    \item for all $a, b, c$ in $A$, if $a \sim b$, we have
    
    \begin{enumerate}[(i)]
    
    \item $a + c \sim b + c$
    \item $c + a \sim c + b$
    \item $a * c \sim b * c$
    \item $c * a \sim c * b$

    \end{enumerate}

    (\vocab{compatibility} with the ring operations $+$ and $*$)

    \end{enumerate}

\end{definition}

\begin{remark}
    \label{mk:FreeAlgebra}

    What we defined here is the \newvocab{free (associative, unital) $R$-algebra} on $X$, it can be denoted $R\langle X \rangle$,
    expressing that it's freely generated by $R$ and $X$, where $X$ is the set of generators.

\end{remark}

\begin{definition}[Linear map]
    \label{LinearMap}
    \leanok
    \lean{LinearMap}
    \uses{Module, RingHom}

    Let $R, S$ be rings, $M$ an $R$-module, $N$ an $S$-module.
    A \newvocab{linear map} from $M$ to $N$ is a function $f : M \to_{l} N$ over a ring homomorphism $\sigma : R \to_{+*} S$, satisfying:

    \begin{enumerate}

    \item $f(x + y) = f(x) + f(y)$ for all $x, y \in M$.
    \item $f(r \bu x) = \sigma(r) \bu f(x)$ for all $r \in R$, $x \in M$.
    
    \end{enumerate}

\end{definition}

\begin{definition}[Tensor algebra]
    \label{TensorAlgebra}
    \leanok
    \lean{TensorAlgebra}
    \uses{FreeAlgebra, LinearMap, RingQuot}

    % Let $R$ be a commutative ring, $M$ a $R$-module.
    Let $A$ be a \vocab{free $R$-algebra} generated by module $M$, let $\iota : M \to A$ denote the map from $M$ to $A$.

    An \newvocab{tensor algebra} over $M$ (or ``of $M$ ") $T$ is the \vocab{ring quotient} of the \vocab{free $R$-algebra} generated by $M$, 
    by the equivalence relation satisfying:

    \begin{enumerate}

    \item for all $a, b$ in $M$, $\iota(a + b) \sim \iota(a) + \iota(b)$.
    \item for all $r$ in $R$, $a$ in $M$, $\iota(r \bu a) \sim r * \iota(a)$.
    
    \end{enumerate}

    i.e. making the inclusion of $M$ into an \vocab{$R$-linear map}.

\end{definition}

\begin{remark}
    \label{mk:TensorAlgebra}

    The definition above is equivalent to the following definition in literature:

    Let $M$ be a module over $R$. An algebra $T$ is called a \vocab{tensor algebra} over $M$ (or ``of $M$ ")
    if it satisfies the following universal property

    \begin{enumerate}
    \item $T$ is an algebra containing $M$ as a \vocab{submodule}, and it is \vocab{generated by} $M$,
    \item Every linear mapping $\lambda$ of $M$ into an algebra $A$ over $R$, can be extended to 
    a \vocab{homomorphism} $\theta$ of $T$ into $A$.
    \end{enumerate}
    
\end{remark}