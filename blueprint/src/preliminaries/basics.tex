
In this section, we follow \cite{jadczyk2019notes}, with supplements from \cite{garling2011clifford, chen2016infinitely}, 
and modifications to match the counterparts in Lean's \Mathlib.

\begin{definition}[Group]
    \label{Group}
    \leanok
    \lean{Group, AddGroup}

    A \vocab{group} is a pair $(G, *)$, where $G$ is a set, $*$ is a binary operation, satisfying:

    \begin{enumerate}
    \item $(g * h) * j = g * (h * j)$ for all $g, h, j \in G$ (\vocab{associativity})
    \item there exists $e$ in $G$ such that $e * g = g * e = g$ for all $g \in G$
    \item for each $g \in G$ there exists $g^{-1} \in G$ such that $g * g^{-1} = g^{-1} * g = e$

    \end{enumerate}

\end{definition}

\begin{remark}
    \label{mk:Group}
    
    It then follows that $e$, the \vocab{identity element}, is unique, and that for each $g \in G$ the \vocab{inverse} $g^{-1}$ is unique.

    A group G is abelian, or \vocab{commutative}, if $g * h = h * g$ for all $g, h \in G$.

\end{remark}

\begin{remark}
    \label{mk:Notation}

    In literatures, the binary operation are usually denoted by juxtaposition, and is understood to be a mapping
    $(g, h) \mapsto g * h$ from $G \times G$ to $G$.
    
    \Mathlib uses a slightly different way to encode this, $G \to G \to G$ is understood to be $G \to (G \to G)$,
    that sends $g \in G$ to a mapping that sends $h \in G$ to $g * h \in G$.
    
    Further more, a mathimatical construct is represented by a "type", as Lean has a dependent type theory foundation. %TODO cite
    
    It can be denoted multiplicatively as $*$ in \MathlibDoc{Group}
    or additively as $+$ in \MathlibDoc{AddGroup}, where $e$ will be denoted by $1$ or $0$, respectively,
    sometimes with subscript (e.g. $1_G$) to indicate where it is.

    We will use the corresponding notation in \Mathlib for future operations without further explanation.

\end{remark}

\begin{definition}[Monoid]
    \label{Monoid}
    \leanok
    \lean{Monoid, AddMonoid}

    A \vocab{monoid} is a pair $(R, *)$, satisfying:

    \begin{enumerate}
    \item $(a * b) * c = a * (b * c)$ for all $a, b, c \in R$ (\vocab{associativity}),

    \item There is an element $1 \in R$ such that $1 * a = a * 1 = a$ for all $a$ in $R$
    (that is $1$ is the multiplicative identity (\vocab{neutral element}).

    \end{enumerate}

\end{definition}

\begin{definition}[Ring]
    \label{Ring}
    \leanok
    \lean{Ring, CommRing, CommSemiring}
    \uses{Group, Monoid}

    A \vocab{ring} is a triple $(R, +, *)$, satisfying:

    \begin{enumerate}
    \item The elements of $R$ form a \vocab{commutative group} under $+$;

    \item The elements of $R$ form a \vocab{monoid} under $*$;

    \item If $a, b, c$ are elements of $R$, we have

    $$
    a * (b + c) = a * b + a * c,
    $$

    $$
    (a + b) * c = a * c + b * c
    $$

    (left and right \vocab{distributivity} over $+$).

    \end{enumerate}

\end{definition}

\begin{definition}[Division ring]
    \label{DivisionRing}
    \leanok
    \lean{DivisionRing}
    \uses{Ring}

    A ring containing at least two elements, in which
    every nonzero element $a$ has a multiplicative inverse $a^{-1}$ is called a \vocab{division ring} 
    (sometimes also called a "skew field").

\end{definition}

% TODO resume later when needed
% \begin{definition}[Field]
%     \label{Field}
%     \leanok
%     \lean{Field}
%     \uses{Ring}

%     A commutative division ring is called a \vocab{field}.

% \end{definition}

\begin{remark}
    \label{mk:CommRing}

    In applications to Clifford algebras $R$ will be always assumed to be \vocab{commutative}.
    
\end{remark}

% TODO resume later when needed
% \begin{definition}[Characteristic]
%     \label{ringChar}
%     \leanok
%     \lean{ringChar}
%     \uses{Ring}

%     Let $R$ be a ring with unit element 1 . The \vocab{characteristic} of $R$ is the smallest positive number $n$ such that

%     $$
%     \underbrace{1+\ldots+1}_{n \text { summands }}=0 \text {. }
%     $$
    
%     If such a number does not exist, the characteristic is defined to be 0 .

% \end{definition}

% \begin{remark}
%     \label{mk:characteristic}

%     Equivalently, it can be defined to be the unique $p \in \mathbb{N}$ satisfying:

%     $$
%     \forall x \in \mathbb{N}, x = 0 \iff p \mid x
%     $$

%     where
    
%     \begin{itemize}

%         \item $p \mid x$ is defined as $\exists y \in \mathbb{N}, x = p y$
    
%         % TODO: confirm to what extent it demands the map
%         \item $x = 0$ asks that there exists a map $f : \mathbb{N} \to R$ such that $0 \in \mathbb{N} ↦ 0 \in R$.
        
%     \end{itemize}

%     This is how the characteristic of $R$ is defined in Lean.

% \end{remark}

\begin{definition}[Module]
    \label{Module}
    \leanok
    \lean{Module}
    \uses{Group, Ring}

    Let $R$ be a commutative ring. A \vocab{module} over $R$ (in short $R$-module) is a set $M$ such that

    \begin{enumerate}
      \item M has a structure of an additive group,
    
      \item For every $a, b \in R$, $x, y \in M$, an operation $\alpha \bu a$ called scalar multiplication is defined, and we have
      
        \begin{enumerate}[i]
            \item $a \bu (x + y) = a \bu x + b \bu y$,
            \item $(a + b) \bu x = a \bu x + b \bu x$,
            \item $a * (b \bu x)=(a * b) \bu x$,
            \item $1_R \bu x = x$.
        \end{enumerate}
    \end{enumerate}

\end{definition}

\begin{remark}
    \label{mk:Module}

    The notation of scalar multiplication is generalized as heterogeneous scalar multiplication in \Mathlib:

    $$
    \alpha \to \beta \to \gamma
    $$

    where $\alpha$, $\beta$, $gamma$ are different types.
    
\end{remark}

\begin{definition}[Vector space]
    \label{VectorSpace}
    \leanok
    \lean{Module}
    \uses{Module, DivisionRing}

    If $R$ is a \vocab{division ring}, then a module $M$ over $R$ is called a \vocab{vector space}.

\end{definition}

\begin{remark}
    \label{mk:VectorSpace}

    For generality, \Mathlib uses \MathlibDoc{Module} throughout for vector spaces,
    particularly, for a vector space $V$, it's usually declared as

    \begin{lstlisting}
        variable [DivisionRing K] [AddCommGroup V] [Module K V]
    \end{lstlisting}

    for definitions/theorems about it, and most of them can be found under \textsf{Mathlib.LinearAlgebra}. % e.g. \MathlibDoc{LinearIndependent}.
    
\end{remark}

\begin{remark}
    \label{mk:Submodule}

    A \vocab{submodule} $N$ of $M$ is a module $N$ such that every element of $N$ is also an element of $M$.

    If $M$ is a vector space then $N$ is called a \vocab{subspace}.

\end{remark}

% TODO resume later when needed
% \begin{definition}[Submodule]
%     \label{Submodule}
%     \leanok
%     \lean{Submodule}
%     \uses{Module}

%     A \vocab{submodule} $N$ of $M$ is a module $N$ such that every element of $N$ is also an element of $M$.

% \end{definition}

% \begin{remark}
%     \label{mk:Submodule}

%     If $M$ is a vector space then $N$ is called a \vocab{subspace}.
    
% \end{remark}

\begin{definition}[Algebra]
    \label{Algebra}
    \leanok
    \lean{Algebra}
    \uses{Module, Ring}

    An \vocab{algebra} $A$ over $R$ is a module over $R$ with a multiplication which makes $A$ a ring and satisfying

    $$
    \alpha(x y)=(\alpha x) y=x(\alpha y),(x, y \in A, \alpha \in R) .
    $$

\end{definition}

\begin{remark}
    \label{mk:Algebra}

    What's simply called algebra is actually associative algebra with identity, a.k.a. \vocab{associative unital algebra}. See
    \href{https://ncatlab.org/nlab/show/red%20herring%20principle}{the red herring principle}
    for more about such phenomenon for naming, particularly the example of (possibly) \vocab{nonassociative algebra}.
    
\end{remark}

\begin{definition}[RingHom]
    \label{RingHom}
    % \leanok
    \lean{RingHom}
    \uses{Ring}

    TODO

\end{definition}

\begin{definition}[FreeAlgebra]
    \label{FreeAlgebra}
    % \leanok
    \lean{FreeAlgebra}
    \uses{Ring, Module}

    TODO

\end{definition}

\begin{definition}[LinearMap]
    \label{LinearMap}
    % \leanok
    \lean{LinearMap}
    \uses{Module, RingHom}

    TODO

\end{definition}

\begin{definition}[RingQuot]
    \label{RingQuot}
    % \leanok
    \lean{RingQuot}
    \uses{Module, RingHom}

    TODO

\end{definition}

\begin{definition}[TensorAlgebra relation]
    \label{TensorAlgebra_Rel}
    % \leanok
    \lean{TensorAlgebra.Rel}
    \uses{FreeAlgebra, LinearMap}

    TODO

\end{definition}

\begin{definition}[Tensor algebra]
    \label{TensorAlgebra}
    % \leanok
    \lean{TensorAlgebra}
    \uses{Algebra, FreeAlgebra, TensorAlgebra_Rel, RingQuot}

    Let $M$ be a module over $R$. An algebra $T$ is called a \vocab{tensor algebra} over $M$ (or "of $M$ ")
    if it satisfies the following universal property

    \begin{enumerate}
    \setcounter{enumi}{1}
    \item $T$ is an algebra containing $M$ as a \vocab{submodule}, and it is \vocab{generated by} $M$,
    \item Every linear mapping $\lambda$ of $M$ into an algebra $A$ over $R$, can be extended to 
    a \vocab{homomorphism} $\theta$ of $T$ into $A$.
    \end{enumerate}

\end{definition}

\begin{remark}
    \label{mk:TensorAlgebra}

    The properties above are equivalent to the following:

    \begin{enumerate}
        \setcounter{enumi}{1}
        \item $T$ is the free (associative, unital) $R$-algebra generated by $M$.
        \item additional relations making the inclusion of $M$ into an $R$-linear map
    \end{enumerate}

    As ideals haven't been formalized for the non-commutative case, \Mathlib uses \MathlibDoc{RingQuot} which is
    the quotient of a non-commutative ring by an arbitrary relation.
    
\end{remark}