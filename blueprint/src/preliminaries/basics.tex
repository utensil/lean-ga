
In this section, we follow \cite{jadczyk2019notes, garling2011clifford}, with modifications to match the counterparts in Lean.

\begin{definition}[Group]
    \label{group}
    \lean{Group}

    A \vocab{group} is a non-empty set $G$ together with a law of composition, a mapping $(g, h) \to gh$ from $G \times G$ to $G$, which satisfies:

    \begin{enumerate}
    \item $(g h) j = g (h j)$ for all $g, h, j \in G$ (\vocab{associativity})
    \item there exists $e$ in $G$ such that $e g = g e = g$ for all $g \in G$
    \item for each $g \in G$ there exists $g^{-1} \in G$ such that $g g^{-1} = g^{-1} g = e$

    \end{enumerate}

\end{definition}

\begin{remark}
    \label{mk:group}
    
    It then follows that $e$, the \vocab{identity element}, is unique, and that for each $g \in G$ the \vocab{inverse} $g^{-1}$ is unique.

    A group G is abelian, or \vocab{commutative}, if $g h = h g$ for all $g, h \in G$.

\end{remark}

\begin{definition}[Monoid]
    \label{monoid}
    \lean{Monoid}

    A \vocab{monoid} is a set $R$ with a law of composition, usually denoted multiplicatively, 
    satisfying the following conditions:

    \begin{enumerate}
    \item $(a b) c = a (b c)$ for all $a, b, c \in R$ (\vocab{associativity}),

    \item There is an element $1 \in R$ such that $1 a = a 1 = a$ for all $a$ in $R$ (that is 1 is the multiplicative identity (\vocab{neutral element}).

    \end{enumerate}

\end{definition}

\begin{definition}[Ring]
    \label{ring}
    \lean{Ring}

    A \vocab{ring} is a set $R$ with two laws of composition, 
    
    \begin{itemize}
        
    \item addition, denoted $+$;
    \item multiplication, denoted by juxtaposition, or \texttt{*} in Lean,
    
    \end{itemize}
    
    which satisfy the following conditions:

    \begin{enumerate}
    \item The elements of $R$ form a \vocab{commutative group} under addition;

    \item The elements of $R$ form a \vocab{monoid} under multiplication;

    \item If $a, b, c$ are elements of $R$, we have

    $$
    a (b + c) = a b + a c, (a + b) c = a c + b c.
    $$

    (left and right \vocab{distributivity} over addition)

    \end{enumerate}



\end{definition}

\begin{remark}
    \label{mk:ring}
    
    A ring containing at least two elements, in which every nonzero element a has a multiplicative inverse $a^{-1}$ is called a \vocab{division ring} (sometimes also called a "skew field").
    A commutative division ring is called a \vocab{field}. 

\end{remark}