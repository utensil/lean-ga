
In this section, we follow \cite{jadczyk2019notes, garling2011clifford}, with modifications to match the counterparts in Lean.

\begin{definition}[Group]
    \label{group}
    \leanok
    \lean{Group}

    A \vocab{group} is a non-empty set $G$ together with a law of composition, a mapping $(g, h) \to gh$ from $G \times G$ to $G$, which satisfies:

    \begin{enumerate}
    \item $(g h) j = g (h j)$ for all $g, h, j \in G$ (\vocab{associativity})
    \item there exists $e$ in $G$ such that $e g = g e = g$ for all $g \in G$
    \item for each $g \in G$ there exists $g^{-1} \in G$ such that $g g^{-1} = g^{-1} g = e$

    \end{enumerate}

\end{definition}

\begin{remark}
    \label{mk:group}
    
    It then follows that $e$, the \vocab{identity element}, is unique, and that for each $g \in G$ the \vocab{inverse} $g^{-1}$ is unique.

    A group G is abelian, or \vocab{commutative}, if $g h = h g$ for all $g, h \in G$.

\end{remark}

\begin{definition}[Monoid]
    \label{monoid}
    \leanok
    \lean{Monoid}

    A \vocab{monoid} is a set $R$ with a law of composition, usually denoted multiplicatively, 
    satisfying the following conditions:

    \begin{enumerate}
    \item $(a b) c = a (b c)$ for all $a, b, c \in R$ (\vocab{associativity}),

    \item There is an element $1 \in R$ such that $1 a = a 1 = a$ for all $a$ in $R$ (that is 1 is the multiplicative identity (\vocab{neutral element}).

    \end{enumerate}

\end{definition}

\begin{definition}[Ring]
    \label{ring}
    \leanok
    \lean{Ring}

    A \vocab{ring} is a set $R$ with two laws of composition, 
    
    \begin{itemize}
        
    \item addition, denoted $+$;
    \item multiplication, denoted by juxtaposition, or \texttt{*} in Lean,
    
    \end{itemize}
    
    which satisfy the following conditions:

    \begin{enumerate}
    \item The elements of $R$ form a \vocab{commutative group} under addition;

    \item The elements of $R$ form a \vocab{monoid} under multiplication;

    \item If $a, b, c$ are elements of $R$, we have

    $$
    a (b + c) = a b + a c, (a + b) c = a c + b c.
    $$

    (left and right \vocab{distributivity} over addition)

    \end{enumerate}

\end{definition}

\begin{definition}[Division ring]
    \label{division_ring}
    \leanok
    \lean{DivisionRing}

    A ring containing at least two elements, in which
    every nonzero element $a$ has a multiplicative inverse $a^{-1}$ is called a \vocab{division ring} 
    (sometimes also called a "skew field").

\end{definition}

\begin{definition}[Field]
    \label{field}
    \leanok
    \lean{Field}

    A commutative division ring is called a \vocab{field}.

\end{definition}

\begin{remark}
    \label{mk:commutative_ring}

    In applications to Clifford algebras $R$ will be always assumed to be \vocab{commutative}.
    
\end{remark}

\begin{definition}[Characteristic]
    \label{characteristic}
    \leanok
    \lean{ringChar}

    Let $R$ be a ring with unit element 1 . The \vocab{characteristic} of $R$ is the smallest positive number $n$ such that

    $$
    \underbrace{1+\ldots+1}_{n \text { summands }}=0 \text {. }
    $$
    
    If such a number does not exist, the characteristic is defined to be 0 .

\end{definition}

\begin{remark}
    \label{mk:characteristic}

    Equivalently, it can be defined to be the unique $p \in \mathbb{N}$ satisfying:

    $$
    \forall x \in \mathbb{N}, x = 0 \iff p \mid x
    $$

    where
    
    \begin{itemize}

        \item $p \mid x$ is defined as $\exists y \in \mathbb{N}, x = p y$
    
        % TODO: confirm to what extent it demands the map
        \item $x = 0$ asks that there exists a map $f : \mathbb{N} \to R$ such that $0 \in \mathbb{N} ↦ 0 \in R$.
        
    \end{itemize}

    This is how the characteristic of $R$ is defined in Lean.

\end{remark}

\begin{definition}[Module]
    \label{module}
    \leanok
    \lean{Module}

    Let $R$ be a commutative ring. A \vocab{module} over $R$ (in short $R$-module) is a set $M$ such that

    \begin{enumerate}
      \item M has a structure of an additive group,
    
      \item For every $\alpha \in R$, $a \in M$ an element $\alpha a \in M$ called scalar multiple is defined, and we have
      
        \begin{enumerate}[i]
            \item $\alpha(x+y)=\alpha x+\alpha y$,
            \item $(\alpha+\beta) x=\alpha x+\beta x$,
            \item $\alpha(\beta x)=(\alpha \beta) x$,
            \item $1 \cdot x=x$.
        \end{enumerate}
    \end{enumerate}

\end{definition}

\begin{definition}[Vector space]
    \label{vector_space}
    \leanok
    \lean{Module}

    If $R$ is a \vocab{division ring}, then a module $M$ over $R$ is called a \vocab{vector space}.

\end{definition}

\begin{remark}
    \label{mk:vector_space}

    In Lean 4, for generality, Mathlib uses \vocab{Module} throughout for vector spaces,
    particularly, for a vector space $V$, it's usually declared as

    \begin{lstlisting}
        variable [DivisionRing K] [AddCommGroup V] [Module K V]
    \end{lstlisting}

    for definitions and theorems about it, and most of them can be found under \texttt{Mathlib.LinearAlgebra} .
    
\end{remark}