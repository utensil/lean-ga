\begin{definition}[Group]
    \label{Group}
    \leanok
    \lean{Group, AddGroup}

    A \newvocab{group} is a pair $(G, *)$, satisfying:

    \begin{enumerate}
    \item $(a * b) * c = a * (b * c)$ for all $a, b, c \in G$ (\vocab{associativity}).
    \item there exists $1 \in G$ such that $1 * a = a * 1 = a$ for all $a \in G$.
    \item for each $a \in G$ there exists $a^{-1} \in G$ such that $a * a^{-1} = a^{-1} * a = 1$.

    \end{enumerate}

\end{definition}

\begin{remark}
    \label{mk:Group}
    
    It then follows that $e$, the \vocab{identity element}, is unique, and that for each $g \in G$ the \vocab{inverse} $g^{-1}$ is unique.

    A group G is abelian, or \vocab{commutative}, if $g * h = h * g$ for all $g, h \in G$.

\end{remark}

\begin{remark}
    \label{mk:Notation}

    In literatures, the binary operation are usually denoted by juxtaposition, and is understood to be a mapping
    $(g, h) \mapsto g * h$ from $G \times G$ to $G$.
    
    \Mathlib uses a slightly different way to encode this, $G \to G \to G$ is understood to be $G \to (G \to G)$,
    that sends $g \in G$ to a mapping that sends $h \in G$ to $g * h \in G$.
    
    Further more, a mathimatical construct is represented by a "type", as Lean has a dependent type theory foundation,
    see \cite{carneiro2019type} and \cite[section 3.2]{ullrich2023extensible}.
    
    It can be denoted multiplicatively as $*$ in \MathlibDoc{Group}
    or additively as $+$ in \MathlibDoc{AddGroup}, where $e$ will be denoted by $1$ or $0$, respectively.

    Sometimes we use notations with subscripts (e.g. $*_G$, $1_G$) to indicate where they are.

    We will use the corresponding notation in \Mathlib for future operations without further explanation.

\end{remark}

\begin{definition}[Monoid]
    \label{Monoid}
    \leanok
    \lean{Monoid, AddMonoid}

    A \newvocab{monoid} is a pair $(R, *)$, satisfying:

    \begin{enumerate}
    \item $(a * b) * c = a * (b * c)$ for all $a, b, c \in R$ (\vocab{associativity}).

    \item there exists an element $1 \in R$ such that $1 * a = a * 1 = a$ for all $a \in R$
    
    i.e. $1$ is the multiplicative identity (\vocab{neutral element}).

    \end{enumerate}

\end{definition}

\begin{definition}[Ring]
    \label{Ring}
    \leanok
    \lean{Ring, CommRing, CommSemiring}
    \uses{Group, Monoid}

    A \newvocab{ring} is a triple $(R, +, *)$, satisfying:

    \begin{enumerate}

    \item $R$ is a \vocab{commutative group} under $+$.

    \item $R$ is a \vocab{monoid} under $*$.

    \item for all $a, b, c \in R$, we have
    
    \begin{enumerate}[(i)]

    \item $a * (b + c) = a * b + a * c$
    \item $(a + b) * c = a * c + b * c$

    \end{enumerate}

    (left and right \vocab{distributivity} over $+$).

    \end{enumerate}

\end{definition}

\begin{remark}
    \label{mk:CommRing}

    In applications to Clifford algebras $R$ will be always assumed to be \vocab{commutative}.
    
\end{remark}

\begin{definition}[Division ring]
    \label{DivisionRing}
    \leanok
    \lean{DivisionRing}
    \uses{Ring}

    A \newvocab{division ring} is a ring $(R, +, *)$, satisfying:

    \begin{enumerate}

    \item $R$ contains at least 2 elements.
    \item for all $a \neq 0$ in $R$, there exists a multiplicative inverse $a^{-1} \in R$ such that
    
    $$
    a * a^{-1} = a^{-1} * a = 1
    $$

    \end{enumerate}

\end{definition}

% TODO resume later when needed
% \begin{definition}[Field]
%     \label{Field}
%     \leanok
%     \lean{Field}
%     \uses{Ring}

%     A commutative division ring is called a \vocab{field}.

% \end{definition}

% TODO resume later when needed
% \begin{definition}[Characteristic]
%     \label{ringChar}
%     \leanok
%     \lean{ringChar}
%     \uses{Ring}

%     Let $R$ be a ring with unit element 1 . The \vocab{characteristic} of $R$ is the smallest positive number $n$ such that

%     $$
%     \underbrace{1+\ldots+1}_{n \text { summands }}=0 \text {. }
%     $$
    
%     If such a number does not exist, the characteristic is defined to be 0 .

% \end{definition}

% \begin{remark}
%     \label{mk:characteristic}

%     Equivalently, it can be defined to be the unique $p \in \mathbb{N}$ satisfying:

%     $$
%     \forall x \in \mathbb{N}, x = 0 \iff p \mid x
%     $$

%     where
    
%     \begin{itemize}

%         \item $p \mid x$ is defined as $\exists y \in \mathbb{N}, x = p y$
    
%         % TODO: confirm to what extent it demands the map
%         \item $x = 0$ asks that there exists a map $f : \mathbb{N} \to R$ such that $0 \in \mathbb{N} ↦ 0 \in R$.
        
%     \end{itemize}

%     This is how the characteristic of $R$ is defined in Lean.

% \end{remark}

\begin{definition}[Module]
    \label{Module}
    \leanok
    \lean{Module}
    \uses{Group, Ring}

    Let $R$ be a commutative ring. A \newvocab{module} over $R$ (in short \vocab{$R$-module}) is a pair $(M, \bu)$, satisfying:

    \begin{enumerate}
      \item M is a group under $+$.
    
      \item $\bu : R \to M \to M$ is called \vocab{scalar multiplication}, for every $a, b \in R$, $x, y \in M$, we have
      
        \begin{enumerate}[(i)]
            \item $a \bu (x + y) = a \bu x + b \bu y$
            \item $(a + b) \bu x = a \bu x + b \bu x$
            \item $a * (b \bu x)=(a * b) \bu x$
            \item $1_R \bu x = x$
        \end{enumerate}
    \end{enumerate}

\end{definition}

\begin{remark}
    \label{mk:Module}

    The notation of scalar multiplication is generalized as heterogeneous scalar multiplication \MathlibDoc{HMul} in \Mathlib:

    $$
    \bu : \alpha \to \beta \to \gamma
    $$

    where $\alpha$, $\beta$, $\gamma$ are different types.
    
\end{remark}

\begin{definition}[Vector space]
    \label{VectorSpace}
    \leanok
    \lean{Module}
    \uses{Module, DivisionRing}

    If $R$ is a \vocab{division ring}, then a module $M$ over $R$ is called a \newvocab{vector space}.

\end{definition}

\begin{remark}
    \label{mk:VectorSpace}

    For generality, \Mathlib uses \MathlibDoc{Module} throughout for vector spaces,
    particularly, for a vector space $V$, it's usually declared as

    \begin{lstlisting}
        variable [DivisionRing K] [AddCommGroup V] [Module K V]
    \end{lstlisting}

    for definitions/theorems about it, and most of them can be found under \textsf{Mathlib.LinearAlgebra} e.g. \MathlibDoc{LinearIndependent}.
    
\end{remark}

\begin{remark}
    \label{mk:Submodule}

    A \newvocab{submodule} $N$ of $M$ is a module $N$ such that every element of $N$ is also an element of $M$.

    If $M$ is a vector space then $N$ is called a \vocab{subspace}.

\end{remark}

% TODO resume later when needed
% \begin{definition}[Submodule]
%     \label{Submodule}
%     \leanok
%     \lean{Submodule}
%     \uses{Module}

%     A \vocab{submodule} $N$ of $M$ is a module $N$ such that every element of $N$ is also an element of $M$.

% \end{definition}

% \begin{remark}
%     \label{mk:Submodule}

%     If $M$ is a vector space then $N$ is called a \vocab{subspace}.
    
% \end{remark}
