% Put any macro and import needed for the project here.
% This will be used by both the web and print versions of the blueprint.
% This file is not meant to be built. Build src/web.tex or src/print.text instead.

% \usepackage{amsmath,amsfonts,amsthm,amssymb}
% \usepackage{hyperref}
% \usepackage{graphicx}
% \DeclareGraphicsExtensions{.svg,.png,.jpg}
% \usepackage[capitalize]{cleveref}
% \usepackage{tikz-cd}
% \usepackage{tikz}

\usepackage[usenames,svgnames,dvipsnames]{xcolor}
\usepackage{amsfonts, amsthm, amssymb, amsmath}
\usepackage{tikz}
\usepackage{tikz-cd}
\usetikzlibrary{shapes,arrows,decorations}
% \usepackage{color} 
% \PassOptionsToPackage{hyphens}{url}
% \usepackage[unicode]{hyperref}
\usepackage[unicode,pagebackref=true,bookmarksopen=true, bookmarksdepth=5]{hyperref}
\setcounter{tocdepth}{5}
\setcounter{secnumdepth}{5}
\hypersetup{
    bookmarksnumbered=true,     
    bookmarksopen=true,         
    bookmarksopenlevel=1, 
    pdfpagemode=UseOutlines
}
\usepackage{cleveref}

% NOTE: Won't work in web version for roman numerals, but it works well for PDF version
\usepackage{enumerate}% http://ctan.org/pkg/enumerate

% FIXME: this fails to work: it breaks at where doi appears
% \usepackage{natbib}

\usepackage{listings}

% \usepackage{verbatim}

% \usepackage[frozencache,cachedir=minted-cache]{minted}
% \usepackage{minted}
% \setminted{
%   fontsize=\footnotesize,
%   frame=single,
%   breaklines
% }
% \setmintedinline{fontsize=\currentfontsize}
% \definecolor{bg}{rgb}{0.95,0.95,0.95}
% \newmintinline[leanc]{theorem.py:LeanLexer -x}{bgcolor=bg}
% \newminted[leancode]{theorem.py:LeanLexer -x}{}
% % https://tex.stackexchange.com/a/276750/41112
% \newlength{\mintedfboxsep}
% \setlength{\mintedfboxsep}{1.5pt}

% handle Lean code highlighting following https://lean-lang.org/lean4/doc/syntax_highlight_in_latex.html
\definecolor{keywordcolor}{rgb}{0.7, 0.1, 0.1}   % red
\definecolor{tacticcolor}{rgb}{0.0, 0.1, 0.6}    % blue
\definecolor{commentcolor}{rgb}{0.4, 0.4, 0.4}   % grey
\definecolor{symbolcolor}{rgb}{0.0, 0.1, 0.6}    % blue
\definecolor{sortcolor}{rgb}{0.1, 0.5, 0.1}      % green
\definecolor{attributecolor}{rgb}{0.7, 0.1, 0.1} % red

\def\lstlanguagefiles{lstlean.tex}
% set default language
\lstset{language=lean}

\date{\today}

\numberwithin{equation}{subsection}

% % Letters
% \newcommand{\C}{\mathbb{C}}
% \newcommand{\bbc}{\mathbb{C}}
% \newcommand{\E}{\mathbb{E}}
% \newcommand*{\bbe}{\mathbb{E}}
% \newcommand{\F}{\mathbb{F}}
% \newcommand{\bbf}{\mathbb{F}}
% \newcommand{\bbH}{\mathbb{H}}
% \newcommand{\bbP}{\mathbb{P}}
% \newcommand{\bbI}{\mathbb{I}}
% \newcommand{\bbn}{\mathbb{N}}
% \newcommand{\bbq}{\mathbb{Q}}
% \newcommand{\bbr}{\mathbb{R}}
% \newcommand{\bbt}{\mathbb{T}}
% \newcommand{\bbz}{\mathbb{Z}}
% \newcommand{\N}{\mathbb{N}}

% \newcommand{\lo}[1]{\mathcal{L}{#1}}

% % Paired delimiters
% \newcommand{\abs}[1]{\left\lvert #1\right\rvert}
% \newcommand{\Abs}[1]{\lvert #1 \rvert}
% \newcommand{\brac}[1]{\left( #1\right)}
% \newcommand{\norm}[1]{\lVert #1\rVert}
% \newcommand{\inn}[1]{\left\langle #1 \right\rangle}

% % Operators
% \DeclareMathOperator{\dist}{dist}

% \newcommand{\ind}[1]{1_{#1}}
% \providecommand{\tup}[1]{{\vec{#1}}}
