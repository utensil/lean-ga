% Those macros are used for the print version of the blueprint.
% This file is not meant to be built. Build src/web.tex or src/print.text instead.

% Define `ifplastex` so we can have complicated macros for PDF
% and simplified macros for the website.
% See http://plastex.github.io/plastex/plastex/sect0006.html
\newif\ifplastex
\plastexfalse

% https://tex.stackexchange.com/a/13268/75671
% \renewcommand\familydefault{\sfdefault}

% https://stackoverflow.com/a/1610161/200764
\hyphenpenalty=750

\hypersetup{%
  colorlinks=true,
  linkcolor={red!50!black},
  citecolor={green!50!black},
  urlcolor={blue!80!black}
}

% Somehow this renders to plain text without brackets, makes it indistinguishable from the surrounding text.
% And it doesn't work with plastex, the citation is not rendered at all for the web version.
% \usepackage{natbib}
% \citestyle{acmauthoryear}

\usepackage[utf8]{inputenc}

% To use mathpazo, we need to use the following lines
% https://mirror-hk.koddos.net/CTAN/macros/latex/required/psnfss/psnfss2e.pdf
% but a lot of missing unicode characters
% \usepackage[T1]{fontenc}
% \usepackage{textcomp}

\usepackage{microtype}

\usepackage{xspace}
\usepackage{xpatch}

% https://tex.stackexchange.com/a/250368/75671
\usepackage{float} %figure inside minipage
\usepackage[autosize,debug]{dot2texi}

% \usepackage[textwidth=14cm]{geometry}
% \usepackage{polyglossia}
% \setdefaultlanguage{english}
% \usepackage{enumitem}
% \usepackage{mathtools}
% \usepackage{thmtools}
% \usepackage{todonotes}

% % ad hoc fix for iota in Lean code
% \setmainfont{latinmodern-math.otf}
% %\setmathfont{Latin Modern Math}
% \setmathfont{latinmodern-math.otf}
% \setmathfont[range=\varnothing]{Asana-Math.otf}
% \setlength{\textwidth}{6.5in}
% \setlength{\oddsidemargin}{-0.1in}
% \setlength{\evensidemargin}{-0.1in}
% \setmathfont[range=\varnothing]{Asana-Math.otf}
% \setmathfont[range=\pitchfork]{Asana-Math.otf}
% \setmathfont[range=\intprod]{Asana-Math.otf}
% \setmathfont[range=\int]{latinmodern-math.otf}

\usepackage{backref}
% Back refs
\renewcommand*{\backrefalt}[4]{%
    \ifcase #1 \footnotesize{(Not cited.)}%
    \or        \footnotesize{(Cited on page~#2)}%
    \else      \footnotesize{(Cited on pages~#2)}%
    \fi}

% theorem styles are implemented by package `mdframed` for LaTeX i.e. PDF
% mostly follows https://github.com/vEnhance/napkin/blob/main/tex/preamble.tex
\usepackage{thmtools}
\usepackage[framemethod=TikZ]{mdframed}

\theoremstyle{definition}
\mdfdefinestyle{mdbluebox}{%
	roundcorner = 10pt,
	linewidth=1pt,
	skipabove=4ex,
	innerbottommargin=9pt,
	skipbelow=2pt,
	nobreak=true,
	linecolor=blue,
	backgroundcolor=TealBlue!5,
}
\declaretheoremstyle[
	headfont=\sffamily\bfseries\color{MidnightBlue},
	mdframed={style=mdbluebox},
	headpunct={\\[3pt]},
	postheadspace={0pt}
]{thmbluebox}

\mdfdefinestyle{mdredbox}{%
	linewidth=0.5pt,
	skipabove=4ex,
	frametitleaboveskip=5pt,
	frametitlebelowskip=0pt,
	skipbelow=2pt,
	frametitlefont=\bfseries,
	innertopmargin=4pt,
	innerbottommargin=8pt,
	nobreak=true,
	linecolor=RawSienna,
	backgroundcolor=Salmon!5,
}
\declaretheoremstyle[
	headfont=\bfseries\color{RawSienna},
	mdframed={style=mdredbox},
	headpunct={\\[3pt]},
	postheadspace={0pt},
]{thmredbox}

\mdfdefinestyle{mdgreenbox}{%
	skipabove=4ex,
	linewidth=2pt,
	rightline=false,
	leftline=true,
	topline=false,
	bottomline=false,
	nobreak=true,
	linecolor=ForestGreen,
	backgroundcolor=ForestGreen!5,
	% innerleftmargin=20pt,
	innerrightmargin=15pt
}
\declaretheoremstyle[
	headfont=\bfseries\sffamily\color{ForestGreen!70!black},
	bodyfont=\normalfont,
	spaceabove=2pt,
	spacebelow=1pt,
	mdframed={style=mdgreenbox},
	headpunct={ --- },
]{thmgreenbox}
\declaretheoremstyle[
	headfont=\bfseries\sffamily\color{ForestGreen!70!black},
	bodyfont=\normalfont,
	spaceabove=2pt,
	spacebelow=1pt,
	mdframed={style=mdgreenbox},
	headpunct={},
]{thmgreenbox*}

\mdfdefinestyle{mdblackbox}{%
	skipabove=4ex,
	linewidth=3pt,
	rightline=false,
	leftline=true,
	topline=false,
	bottomline=false,
	linecolor=black,
	backgroundcolor=RedViolet!5!gray!5,
}
\declaretheoremstyle[
	headfont=\bfseries\sffamily,
	bodyfont=\normalfont\small,
	spaceabove=0pt,
	spacebelow=0pt,
	mdframed={style=mdblackbox}
]{thmblackbox}

\declaretheoremstyle[
	headfont=\bfseries\sffamily,
	bodyfont=\normalfont, %\small, %\sffamily,
	spaceabove=0pt,
	spacebelow=0pt
]{def}

\theoremstyle{definition}
% \declaretheorem[style=thmblackbox,name=Definition,numberwithin=subsection]{definition}
\declaretheorem[style=thmbluebox,name=Theorem,numberwithin=subsection]{theorem}
\declaretheorem[style=thmbluebox,name=Theorem,numberwithin=subsection]{theoremx} % theorem not in dep graph
\declaretheorem[style=thmbluebox,name=Lemma,sibling=theorem]{lemma}
\declaretheorem[style=thmbluebox,name=Lemma,sibling=theorem]{lemmax} % lemma not in dep graph
\declaretheorem[style=thmbluebox,name=Corollary,sibling=theorem]{corollary}
\declaretheorem[style=thmbluebox,name=Proposition,sibling=theorem]{proposition}

\theoremstyle{definition}
% \newtheorem{claim}[theorem]{Claim}
% \newtheorem{definition}[theorem]{Definition}
\declaretheorem[style=def,name=Definition,sibling=theorem]{definition}
% \newtheorem{fact}[theorem]{Fact}
% \newtheorem{abuse}[theorem]{Abuse of Notation}

\declaretheorem[style=thmredbox,name=Example,sibling=theorem]{example}

\theoremstyle{theorem}
% \declaretheorem[name=Question,sibling=theorem,style=thmblackbox]{ques}
% \declaretheorem[name=Exercise,sibling=theorem,style=thmblackbox]{exercise}
\declaretheorem[name=Remark,sibling=theorem,style=thmgreenbox]{remark}
\declaretheorem[name=Remark,sibling=theorem,style=thmgreenbox*]{remark*}
% \declaretheorem[name=Step,style=thmgreenbox]{step} % only used in Lebesgue int

% \declaretheorem[numberwithin=chapter]{theorem}
% \declaretheorem[sibling=theorem]{proposition}
% \declaretheorem[sibling=theorem]{corollary}
% \declaretheorem[sibling=theorem]{remark}
% \declaretheorem[sibling=theorem]{lemma}
% \declaretheorem[sibling=theorem]{definition}
% \declaretheorem[sibling=theorem]{example}

% % We neutralise the Plastex commands
% \newcommand{\uses}[1]{}
% \newcommand{\proves}[1]{}
% \newcommand{\lean}[1]{}
% \newcommand{\leanok}{}

% % Define `ifplastex` so we can have complicated macros for PDF
% % and simplified macros for the website.
% % See http://plastex.github.io/plastex/plastex/sect0006.html
% \newif\ifplastex
% \plastexfalse

% \usepackage[textwidth=14cm]{geometry}
% \usepackage{xfrac}
% \usepackage{polyglossia}
% \setdefaultlanguage{english}

% \usepackage{amsmath,amssymb}
% \usepackage{enumitem}
% \usepackage{hyperref}

% \usepackage{tikz-cd}

% \usepackage{mathtools}
% \usepackage[warnings-off={mathtools-colon,mathtools-overbracket}]{unicode-math}
% \usepackage{fontspec}
% \setmathfont{latinmodern-math.otf}
% \setmathfont[range=\varnothing]{Asana-Math.otf}
% \setmathfont[range=\pitchfork]{Asana-Math.otf}
% \setmathfont[range=\intprod]{Asana-Math.otf}
% \setmathfont[range=\int]{latinmodern-math.otf}

% \usepackage[nameinlink, capitalize]{cleveref}

% \usepackage{amsthm}
% \usepackage{etexcmds}
% \usepackage{thmtools}