\subsection{Clifford algebras - definition}
\label{sec:def}

% \begin{definition}% [Quadratic module]
%   \label{def:quadratic_module}
%   \leanok

%   Let $M$ be a module over a commutative ring $R$, let $Q: V \to R$ be a quadratic form.

%   The pair $(M, Q)$ is a \vocab{quadratic module} (over $R$).
% \end{definition}

% \begin{definition} % [Clifford algebra]
%     \label{def:clifford_algebra}
%     \lean{clifford_algebra}
%     \leanok
%     \uses{dummy}

%     Let $(M, Q)$ be a quadratic module.

%     A \vocab{Clifford algebra} over $M$, denoted $\mathcal{G}(M)$, is
%     the quotient of the tensor algebra $T(M)$
%     by the equivalence relation $v \otimes v \sim Q(v)$.
% \end{definition}

Throughout this section:

Let $M$ be a module over a commutative ring $R$, equipped with a quadratic form $Q: M \to R$.

Let $\iota : M \lmap{R} T(M)$ be the canonical $R$-linear map for the tensor algebra $T(M)$.

Let $\iota_a : R \amap{R} T(M)$ be the canonical map from $R$ to $T(M)$, as a ring homomorphism.

\begin{definition}[Clifford relation]
  \label{CliffordAlgebra_Rel}
  \lean{CliffordAlgebra.Rel}
  % \leanok
  % \uses{Module}

  $\forall m \in M, \iota(m)^2 \sim \iota_a(Q(m))$
\end{definition}

We say that $\iota$ \vocab{is Clifford} if this relation holds.

\begin{definition}[Clifford algebra]
    \label{CliffordAlgebra}
    \lean{CliffordAlgebra}
    % \leanok
    \uses{TensorAlgebra, CliffordAlgebra_Rel}

    A \vocab{Clifford algebra} over $M$, denoted $\Cl (M)$, is
    the quotient of the \vocab{tensor algebra} $T(M)$
    by \vocab{Clifford relation} \ref{CliffordAlgebra_Rel}.
\end{definition}

\begin{remark}
  \label{mk:two_sided_ideals}
  
  In literatures, $M$ is often written $V$, and the quotient is taken by the two-sided ideal $I_Q$ generated from the set
  $\{ v \otimes v - Q(v) \:\vert\: v \in V \}$.

  As of writing, \Mathlib does not have direct support for two-sided ideals,
  but it does support the equivalent operation of taking the quotient by a suitable closure of
  a relation like $v \otimes v \sim Q(v)$.

  Hence the definition above.

\end{remark}

\begin{example}[Clifford algebra over a vector space]
  \label{ex:clifford_algebra_v}

  Let $V$ be a vector space $\RR^n$ over $\RR$, equipped with a quadratic form $Q$.
  
  Since $\RR$ is a commutative ring and $V$ is a module,
  definition \ref{CliffordAlgebra} of Clifford algebra applies.
\end{example}
