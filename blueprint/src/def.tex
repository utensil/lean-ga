\subsection{Clifford algebras - definition}
\label{sec:def}

Let $M$ be a module over a commutative ring $R$, equipped with a quadratic form $Q: M \to R$.

\begin{definition}[Clifford algebra]
    \label{CliffordAlgebra}
    \lean{CliffordAlgebra}
    \leanok
    \uses{TensorAlgebra, QuadraticForm}

    Let $\iota : M \lmap{R} T(M)$ be the canonical $R$-linear map for the tensor algebra $T(M)$.

    Let $\rfun : R \rmap T(M)$ be the canonical map from $R$ to $T(M)$, as a ring homomorphism.

    A \vocab{Clifford algebra} over $Q$, denoted $\Cl (Q)$, is
    the \vocab{ring quotient} of the \vocab{tensor algebra} $T(M)$
    by the equivalence relation satisfying:

    $\forall m \in M, \iota(m)^2 \sim \rfun(Q(m))$.
\end{definition}

\begin{remark}
  \label{mk:two_sided_ideals}
  
  In literatures, $M$ is often written $V$, and the quotient is taken by the two-sided ideal $I_Q$ generated from the set
  $\{ v \otimes v - Q(v) \:\vert\: v \in V \}$.

  As of writing, \Mathlib does not have direct support for two-sided ideals,
  but it does support the equivalent operation of taking the \vocab{ring quotient} by a suitable closure of
  a relation like $v \otimes v \sim Q(v)$.

  Hence the definition above.

\end{remark}

\begin{remark}
  \label{mk:CliffordAlgebra}
  
  This definition and what follows in \Mathlib is initially presented in \cite{wieser2022formalizing},
  some further developments are based on \cite{grinberg2016clifford}, and in turn based on \cite{bourbaki2007}
  which is in French and never translated to English.

  The most informative English reference on \cite{bourbaki2007} is \cite{jadczyk2019notes}, 
  which has an updated exposition in \cite{jadczyk2023bundle}.

\end{remark}

\begin{example}[Clifford algebra over a vector space]
  \label{ex:clifford_algebra_v}

  Let $V$ be a vector space $\RR^n$ over $\RR$, equipped with a quadratic form $Q$.
  
  Since $\RR$ is a commutative ring and $V$ is a module,
  definition \ref{CliffordAlgebra} of Clifford algebra applies.
\end{example}

\begin{definition}[Clifford map]
  \label{iota}
  \lean{CliffordAlgebra.ι}
  \leanok
  \uses{CliffordAlgebra}

  We denote the canonical $R$-linear map to the Clifford algebra $\Cl(M)$ by $\iota : M \lmap{R} \Cl(M)$.
  % , called \newvocab{Clifford map} in some literatures.

\end{definition}

\begin{definition}[Clifford lift]
  \label{lift}
  \lean{CliffordAlgebra.lift}
  \leanok
  \uses{CliffordAlgebra, AlgHom}

  Given a linear map $f : M \lmap{R} A$ into an $R$-algebra $A$, satisfying \forall $m$ in $M$, $f(m)^2 = Q(m)$,
  called \newvocab{is Clifford},
  the canonical \newvocab{lift} of $f$ is defined to be a \vocab{algebra homomorphism} from $\Cl(Q)$ to $A$,
  denoted $\operatorname{lift} f : \Cl(Q) \amap A$.

\end{definition}

\begin{theorem}[The universal property of Clifford algebras]
  \label{univ}
  \lean{CliffordAlgebra.ι_comp_lift, lift_unique}
  \leanok
  \uses{iota, lift}

  Given $f : M \lmap{R} A$, which \vocab{is Clifford}, $F = \operatorname{lift} f$, we have:

  1) $\iota \circ F = f$, i.e. the following diagram commutes:

  \begin{figure}[H]
    \centering
    \begin{tikzcd}[column sep=huge, row sep=huge]
    \Cl(Q) \arrow[r, "F = \operatorname{lift} f"] & A \\
    V \arrow[ru, "f"'] \arrow[u, "\iota"]
    \end{tikzcd}
  \end{figure}

  2) $\operatorname{lift}$ is unique, i.e. given $G : \Cl(Q) \amap A$, 

  $\iota \circ G = f \iff G = \operatorname{lift} f$

\end{theorem}

\begin{remark}
  \label{mk:univ}

The universal property of the Clifford algebras is proven, and should be used instead of the definition
that is subject to change.

\end{remark}